\chapter*{Abstract}
With the development of the 4.0 industrial era, more and more internet-of-thing (IOT) devices have been introduced, which creates a massive amount of data. Thus, mining the hidden valuable insight from these datasets is a very urgent mission. However, data missing is an inevitable problem, especially data taken from IOT devices. The cause of missing is various. It can range from malfunctioned sensors, power outage, data transmission issue to destination server error, which significantly affects the quality of prediction models. Therefore, missing data handling is becoming a notable topic for scientists today. Not only does an appropriate method help to retain essential information, but it also contributes to improve the accuracy of prediction models.

In term of research interest, air quality prediction is one of the problems that cannot be ignored. In recent years, the air quality in Viet Nam is usually worse, especially in metropolises such as Ha Noi, Da Nang and Ho Chi Minh city. Bad air quality has an extreme impact on society, such as resident's health, money, and even socio-economic development. Polluted air is usually caused by fine dust ($PM_{10}$, $PM_{2.5}$, $TSP$), factories and vehicles emission ($SO_2$, $NO_2$, $CO$, $O_3$). Air pollution is a factor that increases respiratory diseases, cardiovascular diseases, and even strokes. As a result, in order to warn and reduce the risk of air pollution, building an forecasting air quality model is extremely important. With this system, residents can prepare preventive measures and minimize the impact of polluted air.

This research focuses on improving the accuracy of air quality prediction model by analyzing and imputing missing data using many sources. It emphasizes the important role of preprocessing data in improving prediction model. Studied methods include traditional methods, deep learning methods, methods using single attribute and methods combining multiple attributes. In addition, the author also conducts experiment with different missing rate to explore how missing rate affects those imputation methods. Finally, preprocessed data is applied to improve the quality of air quality prediction models, which is also the main purpose of this thesis.