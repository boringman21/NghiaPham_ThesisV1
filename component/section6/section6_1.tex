\chapter{Phân tích, huấn luyên và đánh giá hiệu quả trên dữ liệu thực tế}
\section{Tổng quan}\label{ex-1}
...

\begin{figure}[h]
    \centering
    \includegraphics[width=\linewidth]{image/section6_1/experiment.drawio.png}
    \caption{Lược đồ nghiên cứu của thí nghiệm \ref{ex-1}}
    \label{fig:section6_1-experiment}
\end{figure}
Trong thí nghiệm \ref{ex-1}, dữ liệu chỉ số không khí sẽ được xoá ngẫu nhiên để đạt tỉ lệ mất dữ liệu lần lượt là 15\%, 20\% và 30\%. Sau đó, các phương pháp bổ khuyết dữ liệu sẽ được sử dụng để cập nhật các giá trị bị mất. Tập dữ liệu sau khi được xử lý sẽ được so sánh với tập dữ liệu ban đầu để đánh giá độ sai lệch. Kế tiếp, 4 tập dữ liệu được bổ khuyết tốt nhất của mỗi chỉ số trong trường hợp khuyết 20\% dữ liệu sẽ được đem đi huấn luyện các mô hình dự đoán để xem xét mức độ tác động lên chất lượng của các mô hình dự đoán. Hình \ref{fig:section6_1-experiment} mô tả toàn bộ lược đồ thí nghiệm.
...
