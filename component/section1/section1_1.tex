\chapter{Mở đầu}
\section{Thực trạng ô nhiễm không khí}
Ô nhiễm không khí đã và đang là một trong những mối đe doạ lớn nhất đối với sức khoẻ của con người, đặc biệt là trẻ em. Theo Unicef, 99\% dân số toàn cầu đang sinh sống ở những khu vực không đảm bảo về chất lượng không khí. Việc tiếp xúc thường xuyên với không khí ô nhiễm sẽ gây ra tác động lâu dài đến sức khoẻ con người. Ô nhiễm không khí là nguyên nhân gây nên các bệnh đường hô hấp, hen suyễn, đột quỵ và ung thư. Thậm chí, ô nhiễm không khí còn ảnh hưởng đến thai phụ và trẻ sơ sinh. Các hoá chất độc hại từ không khí có thể truyền sang con trong quá trình mang thai và cho con bú. Điều này có thể dẫn đến những hậu quả nghiêm trọng như sảy thai, sinh non, nhẹ cân, thậm chí ảnh hưởng đến sự phát triển não bộ ở trẻ.

\begin{figure}[h]
    \centering
    \includegraphics[width=0.9\textwidth]{image/section1_1/aqi_city.png}
    \caption[Top các địa phương có chỉ số $PM_{2.5}$ cao nhất ở Việt Nam năm 2023]{Top các địa phương có chỉ số $PM_{2.5}$ cao nhất ở Việt Nam năm 2023 \cite{air-quality}}
    \label{fig:sec1_1_aqi_city}
\end{figure}

