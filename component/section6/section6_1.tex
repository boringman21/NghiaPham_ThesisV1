\chapter{Phân tích, huấn luyên và đánh giá hiệu quả trên dữ liệu thực tế}
\section{Tổng quan}\label{ex-1}
...% flowchart LR
%     classDef dataPrep fill:#d4f1f9,stroke:#05445E,stroke-width:2px,color:#05445E,font-weight:bold
%     classDef modelDev fill:#d8f3dc,stroke:#1B4332,stroke-width:2px,color:#1B4332,font-weight:bold
%     classDef anomalyAnalysis fill:#fff1e6,stroke:#9E2A2B,stroke-width:2px,color:#9E2A2B,font-weight:bold
%     classDef evaluation fill:#e9ecef,stroke:#343a40,stroke-width:2px,color:#343a40,font-weight:bold
%     classDef phaseTitle font-weight:bold,font-size:14px
    
%     subgraph Process ["Quy trình phát hiện bất thường"]
%         direction TB
        
%         subgraph DataPrep ["GIAI ĐOẠN 1 CHUẨN BỊ DỮ LIỆU"]
%             direction LR
%             A["1.Thu thập dữ liệu<br>từ các điểm đo"] --> 
%             B["2.Lọc điểm đo<br>đủ 3 kênh dữ liệu"] --> 
%             C["3.Tiền xử lý"]
            
%             subgraph Preprocessing
%                 direction TB
%                 C1["3.1 Chuyển đổi sang<br>DataFrame chuỗi thời gian"]
%                 C2["3.2 Loại bỏ dữ liệu<br>thiếu hoặc lỗi"]
%                 C3["3.3 Tạo chuỗi<br>dữ liệu huấn luyện"]
                
%                 C1 --> C2 --> C3
%             end
            
%             C --- Preprocessing
%         end
        
%         subgraph ModelDev ["GIAI ĐOẠN 2 PHÁT TRIỂN MÔ HÌNH"]
%             direction LR
%             D["4.Phân chia<br>dữ liệu 80/20"] -->
%             E["5.Huấn luyện<br>mô hình dự báo"]
            
%             subgraph Training
%                 direction TB
%                 E1["5.1 Huấn luyện<br>chuyên biệt"]
%                 E2["5.2 Huấn luyện<br>tổng hợp"]
                
%                 E1 --> E2
%             end
            
%             E --- Training -->
%             F["6.Đánh giá<br>hiệu quả mô hình"]
%         end
        
%         subgraph AnomalyDetection ["GIAI ĐOẠN 3 PHÁT HIỆN BẤT THƯỜNG"]
%             direction LR
%             G["7.Phương pháp<br>phát hiện bằng sai số"]
            
%             H["8. Phương pháp<br>học không giám sát"]
            
%             subgraph Methods
%                 direction TB
%                 H1["8.1 Isolation<br>Forest"]
%                 H2["8.2 Local<br>Outlier Factor"]
%                 H3["8.3 One-Class<br>SVM"]
                
%                 H1 --> H2 --> H3
%             end
            
%             G & H --- Methods
%         end
        
%         subgraph ResultAnalysis ["GIAI ĐOẠN 4 PHÂN TÍCH KẾT QUẢ"]
%             I["9 So sánh và đánh giá<br>các phương pháp<br>phát hiện bất thường"]
%         end
%     end
    
%     DataPrep --> ModelDev --> AnomalyDetection --> ResultAnalysis
    
%     subgraph Details ["THÔNG TIN BỔ SUNG"]
%         direction TB
%         N1["Kênh dữ liệu<br>Lưu lượng (chNumber = 1)<br>Áp suất trước (chNumber = 0)<br>Áp suất sau (chNumber = 2)"]
%         N2["Chuỗi huấn luyện<br>Input: 6 giờ dữ liệu liên tục<br>Output: giá trị sau 15 phút"]
%         N3["Mô hình sử dụng<br>LSTM và GRU"]
%         N4["Ngưỡng phát hiện<br>error > 3 x std(error)"]
%     end
    
%     A:::dataPrep
%     B:::dataPrep
%     C:::dataPrep
%     C1:::dataPrep
%     C2:::dataPrep
%     C3:::dataPrep
%     D:::modelDev
%     E:::modelDev
%     E1:::modelDev
%     E2:::modelDev
%     F:::evaluation
%     G:::anomalyAnalysis
%     H:::anomalyAnalysis
%     H1:::anomalyAnalysis
%     H2:::anomalyAnalysis
%     H3:::anomalyAnalysis
%     I:::evaluation
    
%     DataPrep:::phaseTitle
%     ModelDev:::phaseTitle
%     AnomalyDetection:::phaseTitle
%     ResultAnalysis:::phaseTitle
%     Details:::phaseTitle


\begin{figure}[h]
    \centering
    \includegraphics[width=\linewidth]{image/section6_1/experiment.drawio.png}
    \caption{Lược đồ nghiên cứu của thí nghiệm \ref{ex-1}}
    \label{fig:section6_1-experiment}
\end{figure}
Trong thí nghiệm \ref{ex-1}, dữ liệu chỉ số không khí sẽ được xoá ngẫu nhiên để đạt tỉ lệ mất dữ liệu lần lượt là 15\%, 20\% và 30\%. Sau đó, các phương pháp bổ khuyết dữ liệu sẽ được sử dụng để cập nhật các giá trị bị mất. Tập dữ liệu sau khi được xử lý sẽ được so sánh với tập dữ liệu ban đầu để đánh giá độ sai lệch. Kế tiếp, 4 tập dữ liệu được bổ khuyết tốt nhất của mỗi chỉ số trong trường hợp khuyết 20\% dữ liệu sẽ được đem đi huấn luyện các mô hình dự đoán để xem xét mức độ tác động lên chất lượng của các mô hình dự đoán. Hình \ref{fig:section6_1-experiment} mô tả toàn bộ lược đồ thí nghiệm.
...

\section{Mô tả Dữ liệu}

\subsection{Bối cảnh thu thập dữ liệu}

Tập dữ liệu được sử dụng trong nghiên cứu này được thu thập từ một hệ thống giám sát tự động (automatic logger system) triển khai thực tế tại khu vực Bàu Bàng-2, một vùng có hạ tầng cấp thoát nước đang được theo dõi và đánh giá thường xuyên. Hệ thống sử dụng các thiết bị cảm biến điện tử để đo lường các chỉ số kỹ thuật quan trọng tại nhiều điểm đo phân bố trong mạng lưới. Trong phạm vi nghiên cứu này, dữ liệu tập trung vào hai loại chỉ số chính: \textbf{áp suất (Pressure)} và \textbf{lưu lượng (Flow)} — hai thông số nền tảng trong phân tích hiệu suất của hệ thống cấp nước.

Việc thu thập dữ liệu được thực hiện tự động và liên tục, cho phép hình thành các chuỗi thời gian có độ phân giải cao (mỗi 15 phút), từ đó phục vụ các bài toán như phân tích hành vi tiêu thụ, phát hiện rò rỉ, hoặc kiểm soát chất lượng vận hành.

\subsection{Tổ chức và cấu trúc dữ liệu}

Dữ liệu được lưu trữ và cung cấp dưới dạng ba tệp CSV chính: \texttt{find\_query\_1.csv}, \texttt{find\_query\_2.csv}, và \texttt{find\_query\_3.csv}. Mỗi tệp phản ánh thông tin đo được tại các điểm cảm biến trong các khoảng thời gian khác nhau. Các tệp có cấu trúc đồng nhất, trong đó mỗi dòng dữ liệu tương ứng với một phiên ghi dữ liệu tại một kênh cụ thể trong một ngày.

\subsubsection{Các trường dữ liệu chính}

Mỗi bản ghi dữ liệu bao gồm:

\begin{itemize}
    \item \texttt{smsNumber}: Mã định danh duy nhất của thiết bị ghi dữ liệu (logger).
    \item \texttt{chNumber}: Số hiệu kênh đo trên thiết bị. Một thiết bị có thể có nhiều kênh đo (ví dụ: kênh đo áp suất, kênh đo lưu lượng, v.v.).
    \item \texttt{dataType}: Loại dữ liệu được đo (ví dụ: \texttt{Pressure}, \texttt{Flow}).
    \item \texttt{startDate}, \texttt{endDate}: Mốc thời gian bắt đầu và kết thúc phiên đo dữ liệu.
    \item \texttt{dataValues.n.dataTime}, \texttt{dataValues.n.dataValue} (với $n = 0,1,\ldots,95$): Cặp giá trị thời gian - giá trị đo, được ghi nhận đều đặn mỗi 15 phút trong vòng 24 giờ, tạo thành chuỗi thời gian có độ phân giải cao.
    \item \texttt{count}, \texttt{sum}, \texttt{average}, \texttt{min}, \texttt{max}: Các thống kê mô tả của chuỗi giá trị trong ngày.
    \item \texttt{lastDataUpdateTime}: Thời gian cập nhật cuối cùng, cho biết độ mới và độ tin cậy của dữ liệu.
\end{itemize}

\noindent
\textbf{Lưu ý:} Các trường dữ liệu dạng \texttt{dataValues.n.dataTime} và \texttt{dataValues.n.dataValue} được trải rộng dưới dạng các cột riêng biệt (flattened) thay vì mảng lồng nhau như trong JSON.

\vspace{1em}
Hình~\ref{fig:sample_data_structure} dưới đây minh họa cách tổ chức một dòng dữ liệu trong tệp CSV, thể hiện rõ dạng bảng 96 cặp giá trị tương ứng với từng khung giờ trong ngày.

\begin{figure}[htbp]
    \centering
    \includegraphics[width=0.95\textwidth]{image/section6_1/sample_data_structure.png}
    \caption{Cấu trúc dữ liệu một dòng trong tệp \texttt{find\_query\_x.csv}. Mỗi dòng tương ứng với một ngày đo tại một kênh cảm biến.}
    \label{fig:sample_data_structure}
\end{figure}

\subsection{Thông tin ánh xạ kênh đo}

Để xác định rõ chức năng và đơn vị đo của từng kênh, tệp \texttt{channel\_data\_type.csv} được sử dụng như bảng ánh xạ hỗ trợ. Mỗi dòng trong tệp này ánh xạ một thiết bị và kênh cụ thể với các thông tin mô tả kèm theo:

\begin{itemize}
    \item \texttt{Smsnumber}: Trùng với \texttt{smsNumber} trong tệp dữ liệu chính.
    \item \texttt{SerialNumber}: Mã số sản xuất của thiết bị.
    \item \texttt{ChannelNumber}: Số kênh đo (ví dụ: 0, 1, 2).
    \item \texttt{ChannelType}: Loại cảm biến, thường là \texttt{Pressure} hoặc \texttt{Flow}.
    \item \texttt{ChannelUnits}: Đơn vị đo, ví dụ: \texttt{m}, \texttt{l}, \texttt{m\textsuperscript{3}}, hoặc \texttt{mA3}.
\end{itemize}

\noindent
Bảng ánh xạ giúp xác định rõ ý nghĩa dữ liệu từ mỗi kênh đo — ví dụ, kênh số 0 có thể là áp suất tính theo mét cột nước, trong khi kênh số 1 là lưu lượng tính theo lít.

\vspace{0.5em}
Một ví dụ minh họa được trình bày trong Hình~\ref{fig:channel_mapping_table}.

\begin{figure}[htbp]
    \centering
    \includegraphics[width=0.6\textwidth]{image/section6_1/channel_mapping_table.png}
    \caption{Ví dụ bảng ánh xạ từ tệp \texttt{channel\_data\_type.csv}, liên kết giữa thiết bị, kênh đo, và đơn vị.}
    \label{fig:channel_mapping_table}
\end{figure}

\subsection{Đặc điểm kỹ thuật của chuỗi dữ liệu}

Mỗi chuỗi dữ liệu trong một ngày bao gồm 96 giá trị đo, được ghi lại với chu kỳ 15 phút (tức 4 giá trị mỗi giờ). Đây là một chuẩn phổ biến trong hệ thống đo lường cấp thoát nước và điện lực, vì nó cân bằng tốt giữa độ chi tiết và dung lượng lưu trữ. Cấu trúc này rất phù hợp cho các mô hình học máy thời gian thực, mô hình hồi quy chuỗi thời gian (time series regression), hoặc phát hiện bất thường (anomaly detection).

Các thống kê mô tả đi kèm (min, max, trung bình) cho phép kiểm tra sơ bộ chất lượng dữ liệu hoặc phục vụ tiền xử lý như kiểm tra ngưỡng bất thường hoặc các khoảng thời gian bị thiếu dữ liệu.

\subsection{Ý nghĩa cấu trúc kênh đo tại mỗi điểm}

Tùy theo số lượng và loại kênh đo (\texttt{chNumber}) được cấu hình trên từng thiết bị, ta có thể suy ra cấu hình vật lý và chức năng quan trắc tại mỗi điểm đo. Dưới đây là hai mô hình phổ biến:

\begin{itemize}
    \item \textbf{Trường hợp đầy đủ (3 kênh):} Nếu một điểm đo có đủ ba kênh với \texttt{chNumber} = 0, 1, 2 thì có nghĩa:
    \begin{itemize}
        \item Kênh \texttt{chNumber} = 0: Đo áp suất đầu vào (áp suất trước điểm đo).
        \item Kênh \texttt{chNumber} = 1: Đo lưu lượng qua điểm đo.
        \item Kênh \texttt{chNumber} = 2: Đo áp suất đầu ra (áp suất sau điểm đo).
    \end{itemize}
    Cấu hình này thường được áp dụng tại các điểm giao nhận nước, các trạm bơm, hoặc đoạn đường ống quan trọng, nơi cần theo dõi áp suất trước/sau để phát hiện rò rỉ hoặc đánh giá tổn thất áp suất cục bộ.
    
    \item \textbf{Trường hợp tối giản (2 kênh):} Nếu điểm đo chỉ có \texttt{chNumber} = 0 và 1, ta hiểu rằng:
    \begin{itemize}
        \item Kênh 0 vẫn là áp suất (không phân biệt đầu vào/ra).
        \item Kênh 1 là lưu lượng.
    \end{itemize}
    Trường hợp này phổ biến hơn ở các điểm tiêu thụ đơn lẻ (ví dụ: đồng hồ nước hộ gia đình hoặc điểm nhánh nhỏ trong mạng lưới).
\end{itemize}

Nhờ quy ước này, ta có thể phân loại logic cấu trúc điểm đo và mở rộng sang các bài toán đánh giá mạng lưới như phân tích tổn thất áp suất, suy luận chiều dòng chảy, hay xác định vị trí bất thường dựa trên tương quan giữa áp suất và lưu lượng.

% \vspace{0.5em}
% Hình~\ref{fig:channel_logic_structure} dưới đây minh họa trực quan hai cấu hình điển hình này:

% \begin{figure}[H]
%     \centering
%     \includegraphics[width=0.85\textwidth]{images/channel_logic_structure.png}
%     \caption{Cấu hình logic các kênh đo tại điểm: (a) đo đầy đủ đầu vào - lưu lượng - đầu ra; (b) đo áp suất và lưu lượng tổng quát.}
%     \label{fig:channel_logic_structure}
% \end{figure}


\section{Tiền xử lý dữ liệu}
\subsection{Tổng quan}

% TODO: thêm lược đồ tại đây
\subsection{Chuyển đổi dữ liệu}
Quá trình chuyển đổi dữ liệu thô thành định dạng phù hợp cho phân tích và huấn luyện mô hình đóng vai trò then chốt trong tiền xử lý. Dữ liệu thô ban đầu từ hệ thống \texttt{find\_query\_x} cần được tái cấu trúc theo một khuôn mẫu chuẩn hóa nhằm tối ưu hóa cho các bước phân tích tiếp theo. Chúng tôi đã thiết lập và triển khai một quy trình chuyển đổi có hệ thống, được mô tả chi tiết như sau:

\subsubsection{Lọc dữ liệu theo mã thiết bị}
Bước khởi đầu trong quy trình chuyển đổi là thực hiện phân tách dữ liệu theo mã định danh thiết bị (\texttt{smsNumber}). Mỗi thiết bị trong hệ thống được gán một mã định danh duy nhất, và việc phân tách dữ liệu theo thiết bị mang lại nhiều lợi ích quan trọng:

\begin{itemize}
    \item Tạo ra sự tách biệt rõ ràng giữa các nguồn dữ liệu từ các vị trí địa lý khác nhau trong mạng lưới, đảm bảo tính độc lập trong phân tích
    \item Duy trì tính nhất quán và liên tục của dữ liệu trong phạm vi một điểm đo cụ thể, tránh nhiễu từ các nguồn khác
    \item Thiết lập nền tảng cho việc phân tích đặc thù theo từng vị trí, cho phép xây dựng các mô hình chuyên biệt phù hợp với đặc điểm của từng điểm đo
    \item Tạo điều kiện thuận lợi cho việc so sánh và đối chiếu giữa các điểm đo khác nhau trong cùng một hệ thống
\end{itemize}

Quá trình lọc được thực hiện thông qua các phương pháp truy vấn có chọn lọc, trong đó dữ liệu được lọc theo điều kiện mã thiết bị trùng khớp với mã định danh mục tiêu. Kết quả của bước này là một tập dữ liệu con chỉ chứa thông tin từ một thiết bị cụ thể, sẵn sàng cho các bước xử lý tiếp theo.

\subsubsection{Tái cấu trúc dữ liệu theo kênh đo}
Sau khi hoàn tất việc lọc theo thiết bị, dữ liệu thô vẫn tồn tại ở dạng "dài" (long format), trong đó mỗi kênh đo được biểu diễn bằng nhiều bản ghi riêng biệt theo thời gian. Cấu trúc này, mặc dù hiệu quả cho việc lưu trữ, lại không tối ưu cho các phân tích đa biến và huấn luyện mô hình. Do đó, chúng tôi tiến hành chuyển đổi dữ liệu sang định dạng "rộng" (wide format) thông qua một quy trình có hệ thống:

\begin{enumerate}
    \item Nhóm dữ liệu theo các mốc thời gian (\texttt{Timestamp}) để tạo ra cấu trúc thời gian chuẩn
    \item Xác định và trích xuất giá trị từ mỗi kênh đo (\texttt{chNumber}) tại mỗi mốc thời gian
    \item Chuyển đổi các giá trị này thành các cột riêng biệt, tạo ra một cấu trúc ma trận trong đó mỗi hàng đại diện cho một mốc thời gian và mỗi cột đại diện cho một kênh đo
    \item Áp dụng các phép biến đổi bổ sung để đảm bảo tính nhất quán và đầy đủ của dữ liệu
\end{enumerate}

Trong quá trình này, chúng tôi áp dụng một hệ thống ánh xạ chuẩn hóa cho các kênh đo, nhằm tăng tính trực quan và dễ hiểu của dữ liệu:

\begin{itemize}
    \item Kênh 0 (\texttt{chNumber = 0}): Đại diện cho áp suất đầu vào, được gán nhãn là \texttt{Pressure\_1} trong tập dữ liệu đã chuyển đổi
    \item Kênh 1 (\texttt{chNumber = 1}): Đại diện cho lưu lượng, được gán nhãn là \texttt{Flow} trong tập dữ liệu đã chuyển đổi
    \item Kênh 2 (\texttt{chNumber = 2}): Đại diện cho áp suất đầu ra, được gán nhãn là \texttt{Pressure\_2} trong tập dữ liệu đã chuyển đổi
\end{itemize}

Việc ánh xạ này không chỉ tạo ra một cấu trúc dữ liệu rõ ràng mà còn phản ánh chính xác mối quan hệ vật lý giữa các thông số đo lường, tạo điều kiện thuận lợi cho các phân tích thủy lực và phát hiện bất thường trong hệ thống.

\subsubsection{Chuẩn hóa định dạng thời gian}
Việc chuẩn hóa thông tin thời gian đóng vai trò then chốt trong xử lý dữ liệu chuỗi thời gian. Chúng tôi áp dụng định dạng ISO 8601 (YYYY-MM-DD HH:MM:SS) cho cột \texttt{Timestamp}, đảm bảo tính nhất quán và chính xác trong các trường hợp:

\begin{itemize}
    \item Thực hiện các phép tính thời gian phức tạp như xác định khoảng cách giữa các mẫu hoặc phát hiện mẫu bị thiếu
    \item Sắp xếp dữ liệu theo trình tự thời gian chính xác để phân tích xu hướng
\end{itemize}

Quá trình chuẩn hóa bao gồm việc chuyển đổi tất cả biểu diễn thời gian sang định dạng datetime chuẩn và sắp xếp dữ liệu theo thứ tự thời gian tăng dần, tạo nền tảng vững chắc cho các phân tích chuỗi thời gian phức tạp trong các bước tiếp theo.

\subsubsection{Xử lý đơn vị đo}
Để đảm bảo tính nhất quán và khả năng so sánh giữa các điểm đo, chúng tôi thực hiện chuẩn hóa đơn vị đo cho tất cả các giá trị trong tập dữ liệu. Cụ thể, chúng tôi áp dụng các đơn vị chuẩn sau:

\begin{itemize}
    \item \textbf{Áp suất (\texttt{Pressure\_1} và \texttt{Pressure\_2}):} Đơn vị bar, một đơn vị áp suất phổ biến trong ngành cấp thoát nước, tương đương với khoảng 100.000 Pascal
    \item \textbf{Lưu lượng (\texttt{Flow}):} Đơn vị m³/h (mét khối trên giờ), đơn vị tiêu chuẩn để đo lường thể tích chất lỏng chảy qua một điểm trong một đơn vị thời gian
\end{itemize}

Trong trường hợp dữ liệu gốc sử dụng các đơn vị khác (ví dụ: psi cho áp suất hoặc lít/phút cho lưu lượng), chúng tôi áp dụng các hệ số chuyển đổi phù hợp để đảm bảo tính nhất quán trong toàn bộ tập dữ liệu. Việc chuẩn hóa đơn vị đo không chỉ tạo điều kiện thuận lợi cho việc phân tích và so sánh mà còn đảm bảo tính chính xác của các mô hình dự đoán và phát hiện bất thường.

\subsubsection{Kết quả chuyển đổi}
Sau khi hoàn tất quá trình chuyển đổi đa bước nêu trên, chúng tôi thu được một cấu trúc dữ liệu có tổ chức cao, được biểu diễn dưới dạng DataFrame với cấu trúc như sau:

\begin{table}[htbp]
    \centering
    \begin{tabular}{|c|c|c|c|}
        \hline
        \textbf{Timestamp} & \textbf{Pressure\_1} & \textbf{Flow} & \textbf{Pressure\_2} \\
        \hline
        2024-01-01 00:00:00 & 5.2 & 120.5 & 4.8 \\
        2024-01-01 00:15:00 & 5.3 & 118.7 & 4.9 \\
        2024-01-01 00:30:00 & 5.1 & 121.2 & 4.7 \\
        \ldots & \ldots & \ldots & \ldots \\
        \hline
    \end{tabular}
    \caption{Cấu trúc DataFrame sau khi chuyển đổi}
    \label{tab:transformed_data_structure}
\end{table}

DataFrame đã chuyển đổi này có các đặc điểm quan trọng sau:

\begin{itemize}
    \item \textbf{Cấu trúc thời gian nhất quán:} Các mẫu cách đều nhau 15 phút, tạo chuỗi thời gian đồng nhất
    \item \textbf{Tổ chức hợp lý:} Ba cột dữ liệu phản ánh đúng ý nghĩa vật lý của các thông số đo
    \item \textbf{Sắp xếp theo thời gian:} Dữ liệu được sắp xếp tăng dần theo thời gian, thuận lợi cho phân tích xu hướng
    \item \textbf{Tần suất lấy mẫu chuẩn:} Khoảng cách 15 phút giữa các mẫu cho phép phân tích chi tiết biến động trong ngày
\end{itemize}

Cấu trúc này tạo nền tảng vững chắc cho việc phân tích và huấn luyện mô hình.

\subsection{Chọn dữ liệu để huấn luyện mô hình}
Trong quá trình nghiên cứu, việc lựa chọn tập dữ liệu phù hợp đóng vai trò quyết định đến hiệu suất và độ tin cậy của mô hình học máy. Sau khi phân tích cấu trúc và đặc điểm của bộ dữ liệu, chúng tôi quyết định tập trung vào các điểm đo có đầy đủ ba kênh đo (\texttt{chNumber} = 0, 1, 2) với các lý do sau:

\subsubsection{Tính đầy đủ của thông tin vật lý}
Các điểm đo có đủ ba kênh cung cấp bức tranh toàn diện về trạng thái thủy lực tại mỗi vị trí. Cụ thể, việc đồng thời có thông tin về áp suất đầu vào, lưu lượng, và áp suất đầu ra cho phép:

\begin{itemize}
    \item Tính toán chênh lệch áp suất ($\Delta P = P_{in} - P_{out}$) trực tiếp, là chỉ số quan trọng để phát hiện tổn thất áp suất bất thường.
    \item Thiết lập mối tương quan giữa lưu lượng và chênh lệch áp suất, tuân theo định luật Darcy-Weisbach và các nguyên lý thủy lực cơ bản.
    \item Xây dựng các đặc trưng phái sinh (derived features) có giá trị cao trong việc phát hiện rò rỉ, như hệ số tổn thất cục bộ, chỉ số biến thiên áp suất theo lưu lượng, v.v.
\end{itemize}

\subsubsection{Khả năng phát hiện bất thường đa chiều}
Với ba kênh đo, mô hình có thể phát hiện bất thường theo nhiều chiều và dạng biểu hiện khác nhau:

\begin{itemize}
    \item \textbf{Bất thường về áp suất đầu vào:} Có thể phản ánh vấn đề từ nguồn cung cấp hoặc đoạn ống phía trước.
    \item \textbf{Bất thường về lưu lượng:} Phát hiện tiêu thụ bất thường, rò rỉ lớn, hoặc sự cố van.
    \item \textbf{Bất thường về áp suất đầu ra:} Phản ánh vấn đề ở phía hạ nguồn hoặc tình trạng tắc nghẽn.
    \item \textbf{Bất thường về mối tương quan:} Quan trọng nhất, có thể phát hiện rò rỉ nhỏ thông qua sự thay đổi trong mối quan hệ giữa ba thông số, ngay cả khi mỗi thông số riêng lẻ vẫn nằm trong ngưỡng bình thường.
\end{itemize}

\subsubsection{Tính ổn định và độ tin cậy của mô hình}
Mô hình được huấn luyện trên dữ liệu đầy đủ ba kênh sẽ có độ tin cậy cao hơn vì:

\begin{itemize}
    \item Giảm thiểu sự phụ thuộc vào một kênh đo duy nhất, từ đó giảm tác động của nhiễu cục bộ hoặc lỗi cảm biến.
    \item Cho phép áp dụng các kỹ thuật kiểm tra chéo (cross-validation) giữa các kênh để xác nhận tính hợp lệ của dữ liệu.
    \item Tạo điều kiện cho việc áp dụng các phương pháp học sâu phức tạp hơn như mạng nơ-ron tích chập (CNN) hoặc mạng LSTM đa đầu vào, vốn hoạt động hiệu quả với dữ liệu đa chiều.
\end{itemize}

\subsubsection{Phương pháp lựa chọn cụ thể}
Dựa trên các lý do trên, chúng tôi áp dụng quy trình lựa chọn dữ liệu như sau:

\begin{enumerate}
    \item Quét toàn bộ tập dữ liệu để xác định các điểm đo có đủ ba kênh (\texttt{chNumber} = 0, 1, 2).
    \item Đối với mỗi điểm đo được chọn, kiểm tra tính đầy đủ của dữ liệu theo thời gian.
\end{enumerate}

Kết quả của quá trình lựa chọn này là một tập con gồm $N$ điểm đo đáp ứng đầy đủ các tiêu chí, sẽ được sử dụng làm cơ sở cho các bước tiền xử lý và huấn luyện mô hình tiếp theo.
\subsubsection{Phân tích các dữ liệu tại từng điểm đo}
Sau khi tiến hành phân tích toàn diện, chúng tôi đã tổng hợp thông tin về tình trạng dữ liệu tại các điểm đo trong bảng dưới đây. Bảng \ref{tab:data-overview} trình bày tổng quan về các đặc điểm quan trọng của từng điểm đo, bao gồm mã định danh, khung thời gian đo, tình trạng dữ liệu lưu lượng và mức độ khuyết thiếu dữ liệu.

\begin{table}[htbp]
\centering
\resizebox{\textwidth}{!}{%
\begin{tabular}{|c|c|c|c|c|c|}
\hline
\textbf{smsNumber} & \textbf{Khung thời gian} & \textbf{Tình trạng lưu lượng} & \textbf{Mức độ khuyết thiếu} & \textbf{Số kênh đo} & \textbf{Ghi chú} \\
\hline
SMS001 & 15 phút & Đầy đủ & Thấp (<5\%) & 3 kênh & Dữ liệu chất lượng cao \\
\hline
SMS002 & 15 phút & Không đầy đủ & Cao (>30\%) & 3 kênh & Nhiều đoạn dữ liệu bị mất \\
\hline
SMS003 & 5 phút & Đầy đủ & Trung bình (10-15\%) & 3 kênh & Tần suất đo cao \\
\hline
SMS004 & 15 phút & Đầy đủ & Thấp (<5\%) & 3 kênh & Dữ liệu chất lượng cao \\
\hline
SMS005 & 5 phút & Không đầy đủ & Rất cao (>50\%) & 2 kênh & Thiếu kênh áp suất đầu ra \\
\hline
SMS006 & 15 phút & Đầy đủ & Trung bình (10-15\%) & 3 kênh & Một số đoạn nhiễu \\
\hline
SMS007 & 5 phút & Không đầy đủ & Cao (>30\%) & 3 kênh & Dữ liệu không liên tục \\
\hline
SMS008 & 15 phút & Đầy đủ & Thấp (<5\%) & 3 kênh & Dữ liệu chất lượng cao \\
\hline
\end{tabular}%
}
\caption{Tổng quan đặc điểm dữ liệu tại các điểm đo}
\label{tab:data-overview}
\end{table}

Từ bảng tổng quan này, có thể thấy rằng một số điểm đo như SMS001, SMS004 và SMS008 có chất lượng dữ liệu tốt với mức độ khuyết thiếu thấp và dữ liệu lưu lượng đầy đủ. Ngược lại, các điểm đo như SMS002, SMS005 và SMS007 có tỷ lệ khuyết thiếu cao, đặc biệt là SMS005 với hơn 50\% dữ liệu bị thiếu và chỉ có 2 kênh đo, gây khó khăn đáng kể cho quá trình phân tích và mô hình hóa.

Các điểm đo thiếu dữ liệu

Các điểm đo có đủ dữ liệu để xử lý


\subsection{Xử lý dữ liệu thiếu}
\subsection{Feature selection}
\subsection{Tạo chuỗi dữ liệu huấn luyện}
\subsection{Tạo tập dữ liệu huấn luyện và kiểm tra}

\section{Huấn luyện mô hình}
\subsection{Tổng quan}
\subsection{Huấn luyện mô hình}
\subsection{Đánh giá mô hình}

\section{Phát hiện bất thường}
\subsection{Tổng quan}
\subsection{Phát hiện bằng sai số}
\subsection{Phát hiện bằng các phương pháp học không giám sát}
\subsubsection{Isolation Forest}
\subsubsection{One-class SVM}
\subsubsection{LOF}
\subsubsection{One-class SVM}
\subsection{Phân tích kết quả}


\section{Kết luận}

