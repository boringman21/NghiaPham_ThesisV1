\section{Kết luận}

Chương này đã trình bày một quy trình toàn diện về phát hiện bất thường trong hệ thống cấp nước, từ việc thu thập và tiền xử lý dữ liệu thực tế đến huấn luyện mô hình và triển khai các phương pháp phát hiện bất thường. Thông qua nghiên cứu này, chúng tôi đã đạt được những kết quả quan trọng sau:

Đầu tiên, dữ liệu thực tế từ hệ thống giám sát tự động tại khu vực Bàu Bàng (Bình Dương) đã được xử lý và chuyển đổi thành định dạng phù hợp cho việc phân tích. Quy trình tiền xử lý bao gồm năm giai đoạn chính: chuyển đổi dữ liệu, chọn dữ liệu, xử lý dữ liệu thiếu, lựa chọn đặc trưng và tạo dữ liệu huấn luyện. Đáng chú ý, việc lựa chọn các điểm đo có đủ ba kênh (áp suất đầu vào, lưu lượng và áp suất đầu ra) đã tạo nền tảng vững chắc cho việc phát hiện bất thường đa chiều, mặc dù nhiều điểm đo gặp phải vấn đề về tính đầy đủ của dữ liệu.

Về mặt huấn luyện mô hình, nghiên cứu đã triển khai hai kiến trúc mạng nơ-ron hồi quy là LSTM và GRU với hai chiến lược huấn luyện khác nhau: mô hình riêng cho từng điểm đo và mô hình chung cho tất cả điểm đo. Kết quả cho thấy mô hình chung có hiệu suất tốt hơn đáng kể đối với điểm đo 841211914190, trong khi hiệu suất tương đương ở điểm đo 8401210607558. Việc bổ sung dữ liệu áp suất vào quá trình huấn luyện đã cải thiện hiệu suất ở một số trường hợp, minh chứng cho giá trị của việc kết hợp nhiều kênh dữ liệu trong phân tích.

Trong phần phát hiện bất thường, hai phương pháp chính đã được áp dụng: phát hiện bằng sai số dựa trên quy tắc 3-sigma và phát hiện bằng học không giám sát với các thuật toán như Isolation Forest, Local Outlier Factor và One-class SVM. Kết quả phát hiện bất thường đã xác định các điểm dữ liệu đáng ngờ, đặc biệt trong các khoảng thời gian có biến động lớn về lưu lượng và vào ban đêm khi tiêu thụ nước thường ổn định. Đáng chú ý, các mô hình khác nhau đã phát hiện các mẫu bất thường khác nhau, cho thấy giá trị của việc kết hợp nhiều phương pháp để có cái nhìn toàn diện hơn.

Nghiên cứu này đã minh chứng tính khả thi và hiệu quả của việc áp dụng các kỹ thuật học máy và phát hiện bất thường trên dữ liệu thực tế từ hệ thống cấp nước. Các mô hình được huấn luyện không chỉ có khả năng dự báo lưu lượng nước với độ chính xác cao mà còn phát hiện các bất thường tiềm ẩn, đóng góp đáng kể vào việc nâng cao hiệu quả quản lý và vận hành hệ thống cấp nước. Những kết quả này tạo nền tảng cho việc xây dựng các hệ thống giám sát thông minh, giúp phát hiện sớm các sự cố, giảm thiểu thất thoát nước và tối ưu hóa chi phí vận hành.

Tuy nhiên, nghiên cứu cũng gặp phải một số thách thức như vấn đề dữ liệu thiếu và tính đặc thù của từng điểm đo. Những thách thức này mở ra hướng nghiên cứu trong tương lai về việc phát triển các phương pháp xử lý dữ liệu thiếu hiệu quả hơn và các mô hình có khả năng thích ứng với đặc điểm cụ thể của từng điểm đo trong hệ thống cấp nước.