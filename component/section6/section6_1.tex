\chapter{Thực nghiệm dữ liệu thực tế và đánh giá}
\section{Bổ khuyết và huấn luyện mô hình dự đoán chất lượng không khí dựa trên mất mát dữ liệu nhân tạo trên tập quan trắc}\label{ex-1}
\subsection{Tổng quan}
% flowchart LR
%     classDef dataPrep fill:#d4f1f9,stroke:#05445E,stroke-width:2px,color:#05445E,font-weight:bold
%     classDef modelDev fill:#d8f3dc,stroke:#1B4332,stroke-width:2px,color:#1B4332,font-weight:bold
%     classDef anomalyAnalysis fill:#fff1e6,stroke:#9E2A2B,stroke-width:2px,color:#9E2A2B,font-weight:bold
%     classDef evaluation fill:#e9ecef,stroke:#343a40,stroke-width:2px,color:#343a40,font-weight:bold
%     classDef phaseTitle font-weight:bold,font-size:14px
    
%     subgraph Process ["Quy trình phát hiện bất thường"]
%         direction TB
        
%         subgraph DataPrep ["GIAI ĐOẠN 1 CHUẨN BỊ DỮ LIỆU"]
%             direction LR
%             A["1.Thu thập dữ liệu<br>từ các điểm đo"] --> 
%             B["2.Lọc điểm đo<br>đủ 3 kênh dữ liệu"] --> 
%             C["3.Tiền xử lý"]
            
%             subgraph Preprocessing
%                 direction TB
%                 C1["3.1 Chuyển đổi sang<br>DataFrame chuỗi thời gian"]
%                 C2["3.2 Loại bỏ dữ liệu<br>thiếu hoặc lỗi"]
%                 C3["3.3 Tạo chuỗi<br>dữ liệu huấn luyện"]
                
%                 C1 --> C2 --> C3
%             end
            
%             C --- Preprocessing
%         end
        
%         subgraph ModelDev ["GIAI ĐOẠN 2 PHÁT TRIỂN MÔ HÌNH"]
%             direction LR
%             D["4.Phân chia<br>dữ liệu 80/20"] -->
%             E["5.Huấn luyện<br>mô hình dự báo"]
            
%             subgraph Training
%                 direction TB
%                 E1["5.1 Huấn luyện<br>chuyên biệt"]
%                 E2["5.2 Huấn luyện<br>tổng hợp"]
                
%                 E1 --> E2
%             end
            
%             E --- Training -->
%             F["6.Đánh giá<br>hiệu quả mô hình"]
%         end
        
%         subgraph AnomalyDetection ["GIAI ĐOẠN 3 PHÁT HIỆN BẤT THƯỜNG"]
%             direction LR
%             G["7.Phương pháp<br>phát hiện bằng sai số"]
            
%             H["8. Phương pháp<br>học không giám sát"]
            
%             subgraph Methods
%                 direction TB
%                 H1["8.1 Isolation<br>Forest"]
%                 H2["8.2 Local<br>Outlier Factor"]
%                 H3["8.3 One-Class<br>SVM"]
                
%                 H1 --> H2 --> H3
%             end
            
%             G & H --- Methods
%         end
        
%         subgraph ResultAnalysis ["GIAI ĐOẠN 4 PHÂN TÍCH KẾT QUẢ"]
%             I["9 So sánh và đánh giá<br>các phương pháp<br>phát hiện bất thường"]
%         end
%     end
    
%     DataPrep --> ModelDev --> AnomalyDetection --> ResultAnalysis
    
%     subgraph Details ["THÔNG TIN BỔ SUNG"]
%         direction TB
%         N1["Kênh dữ liệu<br>Lưu lượng (chNumber = 1)<br>Áp suất trước (chNumber = 0)<br>Áp suất sau (chNumber = 2)"]
%         N2["Chuỗi huấn luyện<br>Input: 6 giờ dữ liệu liên tục<br>Output: giá trị sau 15 phút"]
%         N3["Mô hình sử dụng<br>LSTM và GRU"]
%         N4["Ngưỡng phát hiện<br>error > 3 x std(error)"]
%     end
    
%     A:::dataPrep
%     B:::dataPrep
%     C:::dataPrep
%     C1:::dataPrep
%     C2:::dataPrep
%     C3:::dataPrep
%     D:::modelDev
%     E:::modelDev
%     E1:::modelDev
%     E2:::modelDev
%     F:::evaluation
%     G:::anomalyAnalysis
%     H:::anomalyAnalysis
%     H1:::anomalyAnalysis
%     H2:::anomalyAnalysis
%     H3:::anomalyAnalysis
%     I:::evaluation
    
%     DataPrep:::phaseTitle
%     ModelDev:::phaseTitle
%     AnomalyDetection:::phaseTitle
%     ResultAnalysis:::phaseTitle
%     Details:::phaseTitle


\begin{figure}[h]
    \centering
    \includegraphics[width=\linewidth]{image/section6_1/experiment.drawio.png}
    \caption{Lược đồ nghiên cứu của thí nghiệm \ref{ex-1}}
    \label{fig:section6_1-experiment}
\end{figure}
Trong thí nghiệm \ref{ex-1}, dữ liệu chỉ số không khí sẽ được xoá ngẫu nhiên để đạt tỉ lệ mất dữ liệu lần lượt là 15\%, 20\% và 30\%. Sau đó, các phương pháp bổ khuyết dữ liệu sẽ được sử dụng để cập nhật các giá trị bị mất. Tập dữ liệu sau khi được xử lý sẽ được so sánh với tập dữ liệu ban đầu để đánh giá độ sai lệch. Kế tiếp, 4 tập dữ liệu được bổ khuyết tốt nhất của mỗi chỉ số trong trường hợp khuyết 20\% dữ liệu sẽ được đem đi huấn luyện các mô hình dự đoán để xem xét mức độ tác động lên chất lượng của các mô hình dự đoán. Hình \ref{fig:section6_1-experiment} mô tả toàn bộ lược đồ thí nghiệm.
...

\section{Mô tả Dữ liệu}

\subsection{Bối cảnh thu thập dữ liệu}

Tập dữ liệu được sử dụng trong nghiên cứu này được thu thập từ một hệ thống giám sát tự động (automatic logger system) triển khai thực tế tại khu vực Bàu Bàng-2, một vùng có hạ tầng cấp thoát nước đang được theo dõi và đánh giá thường xuyên. Hệ thống sử dụng các thiết bị cảm biến điện tử để đo lường các chỉ số kỹ thuật quan trọng tại nhiều điểm đo phân bố trong mạng lưới. Trong phạm vi nghiên cứu này, dữ liệu tập trung vào hai loại chỉ số chính: \textbf{áp suất (Pressure)} và \textbf{lưu lượng (Flow)} — hai thông số nền tảng trong phân tích hiệu suất của hệ thống cấp nước.

Việc thu thập dữ liệu được thực hiện tự động và liên tục, cho phép hình thành các chuỗi thời gian có độ phân giải cao (mỗi 15 phút), từ đó phục vụ các bài toán như phân tích hành vi tiêu thụ, phát hiện rò rỉ, hoặc kiểm soát chất lượng vận hành.

\subsection{Tổ chức và cấu trúc dữ liệu}

Dữ liệu được lưu trữ và cung cấp dưới dạng ba tệp CSV chính: \texttt{find\_query\_1.csv}, \texttt{find\_query\_2.csv}, và \texttt{find\_query\_3.csv}. Mỗi tệp phản ánh thông tin đo được tại các điểm cảm biến trong các khoảng thời gian khác nhau. Các tệp có cấu trúc đồng nhất, trong đó mỗi dòng dữ liệu tương ứng với một phiên ghi dữ liệu tại một kênh cụ thể trong một ngày.

\subsubsection{Các trường dữ liệu chính}

Mỗi bản ghi dữ liệu bao gồm:

\begin{itemize}
    \item \texttt{smsNumber}: Mã định danh duy nhất của thiết bị ghi dữ liệu (logger).
    \item \texttt{chNumber}: Số hiệu kênh đo trên thiết bị. Một thiết bị có thể có nhiều kênh đo (ví dụ: kênh đo áp suất, kênh đo lưu lượng, v.v.).
    \item \texttt{dataType}: Loại dữ liệu được đo (ví dụ: \texttt{Pressure}, \texttt{Flow}).
    \item \texttt{startDate}, \texttt{endDate}: Mốc thời gian bắt đầu và kết thúc phiên đo dữ liệu.
    \item \texttt{dataValues.n.dataTime}, \texttt{dataValues.n.dataValue} (với $n = 0,1,\ldots,95$): Cặp giá trị thời gian - giá trị đo, được ghi nhận đều đặn mỗi 15 phút trong vòng 24 giờ, tạo thành chuỗi thời gian có độ phân giải cao.
    \item \texttt{count}, \texttt{sum}, \texttt{average}, \texttt{min}, \texttt{max}: Các thống kê mô tả của chuỗi giá trị trong ngày.
    \item \texttt{lastDataUpdateTime}: Thời gian cập nhật cuối cùng, cho biết độ mới và độ tin cậy của dữ liệu.
\end{itemize}

\noindent
\textbf{Lưu ý:} Các trường dữ liệu dạng \texttt{dataValues.n.dataTime} và \texttt{dataValues.n.dataValue} được trải rộng dưới dạng các cột riêng biệt (flattened) thay vì mảng lồng nhau như trong JSON.

\vspace{1em}
Hình~\ref{fig:sample_data_structure} dưới đây minh họa cách tổ chức một dòng dữ liệu trong tệp CSV, thể hiện rõ dạng bảng 96 cặp giá trị tương ứng với từng khung giờ trong ngày.

\begin{figure}[htbp]
    \centering
    \includegraphics[width=0.95\textwidth]{image/section6_1/sample_data_structure.png}
    \caption{Cấu trúc dữ liệu một dòng trong tệp \texttt{find\_query\_x.csv}. Mỗi dòng tương ứng với một ngày đo tại một kênh cảm biến.}
    \label{fig:sample_data_structure}
\end{figure}

\subsection{Thông tin ánh xạ kênh đo}

Để xác định rõ chức năng và đơn vị đo của từng kênh, tệp \texttt{channel\_data\_type.csv} được sử dụng như bảng ánh xạ hỗ trợ. Mỗi dòng trong tệp này ánh xạ một thiết bị và kênh cụ thể với các thông tin mô tả kèm theo:

\begin{itemize}
    \item \texttt{Smsnumber}: Trùng với \texttt{smsNumber} trong tệp dữ liệu chính.
    \item \texttt{SerialNumber}: Mã số sản xuất của thiết bị.
    \item \texttt{ChannelNumber}: Số kênh đo (ví dụ: 0, 1, 2).
    \item \texttt{ChannelType}: Loại cảm biến, thường là \texttt{Pressure} hoặc \texttt{Flow}.
    \item \texttt{ChannelUnits}: Đơn vị đo, ví dụ: \texttt{m}, \texttt{l}, \texttt{m\textsuperscript{3}}, hoặc \texttt{mA3}.
\end{itemize}

\noindent
Bảng ánh xạ giúp xác định rõ ý nghĩa dữ liệu từ mỗi kênh đo — ví dụ, kênh số 0 có thể là áp suất tính theo mét cột nước, trong khi kênh số 1 là lưu lượng tính theo lít.

\vspace{0.5em}
Một ví dụ minh họa được trình bày trong Hình~\ref{fig:channel_mapping_table}.

\begin{figure}[htbp]
    \centering
    \includegraphics[width=0.6\textwidth]{image/section6_1/channel_mapping_table.png}
    \caption{Ví dụ bảng ánh xạ từ tệp \texttt{channel\_data\_type.csv}, liên kết giữa thiết bị, kênh đo, và đơn vị.}
    \label{fig:channel_mapping_table}
\end{figure}

\subsection{Đặc điểm kỹ thuật của chuỗi dữ liệu}

Mỗi chuỗi dữ liệu trong một ngày bao gồm 96 giá trị đo, được ghi lại với chu kỳ 15 phút (tức 4 giá trị mỗi giờ). Đây là một chuẩn phổ biến trong hệ thống đo lường cấp thoát nước và điện lực, vì nó cân bằng tốt giữa độ chi tiết và dung lượng lưu trữ. Cấu trúc này rất phù hợp cho các mô hình học máy thời gian thực, mô hình hồi quy chuỗi thời gian (time series regression), hoặc phát hiện bất thường (anomaly detection).

Các thống kê mô tả đi kèm (min, max, trung bình) cho phép kiểm tra sơ bộ chất lượng dữ liệu hoặc phục vụ tiền xử lý như kiểm tra ngưỡng bất thường hoặc các khoảng thời gian bị thiếu dữ liệu.

\subsection{Ý nghĩa cấu trúc kênh đo tại mỗi điểm}

Tùy theo số lượng và loại kênh đo (\texttt{chNumber}) được cấu hình trên từng thiết bị, ta có thể suy ra cấu hình vật lý và chức năng quan trắc tại mỗi điểm đo. Dưới đây là hai mô hình phổ biến:

\begin{itemize}
    \item \textbf{Trường hợp đầy đủ (3 kênh):} Nếu một điểm đo có đủ ba kênh với \texttt{chNumber} = 0, 1, 2 thì có nghĩa:
    \begin{itemize}
        \item Kênh \texttt{chNumber} = 0: Đo áp suất đầu vào (áp suất trước điểm đo).
        \item Kênh \texttt{chNumber} = 1: Đo lưu lượng qua điểm đo.
        \item Kênh \texttt{chNumber} = 2: Đo áp suất đầu ra (áp suất sau điểm đo).
    \end{itemize}
    Cấu hình này thường được áp dụng tại các điểm giao nhận nước, các trạm bơm, hoặc đoạn đường ống quan trọng, nơi cần theo dõi áp suất trước/sau để phát hiện rò rỉ hoặc đánh giá tổn thất áp suất cục bộ.
    
    \item \textbf{Trường hợp tối giản (2 kênh):} Nếu điểm đo chỉ có \texttt{chNumber} = 0 và 1, ta hiểu rằng:
    \begin{itemize}
        \item Kênh 0 vẫn là áp suất (không phân biệt đầu vào/ra).
        \item Kênh 1 là lưu lượng.
    \end{itemize}
    Trường hợp này phổ biến hơn ở các điểm tiêu thụ đơn lẻ (ví dụ: đồng hồ nước hộ gia đình hoặc điểm nhánh nhỏ trong mạng lưới).
\end{itemize}

Nhờ quy ước này, ta có thể phân loại logic cấu trúc điểm đo và mở rộng sang các bài toán đánh giá mạng lưới như phân tích tổn thất áp suất, suy luận chiều dòng chảy, hay xác định vị trí bất thường dựa trên tương quan giữa áp suất và lưu lượng.

% \vspace{0.5em}
% Hình~\ref{fig:channel_logic_structure} dưới đây minh họa trực quan hai cấu hình điển hình này:

% \begin{figure}[H]
%     \centering
%     \includegraphics[width=0.85\textwidth]{images/channel_logic_structure.png}
%     \caption{Cấu hình logic các kênh đo tại điểm: (a) đo đầy đủ đầu vào - lưu lượng - đầu ra; (b) đo áp suất và lưu lượng tổng quát.}
%     \label{fig:channel_logic_structure}
% \end{figure}


\section{Tiền xử lý dữ liệu}
\subsection{Tổng quan}
\subsection{Chuyển đổi dữ liệu}
\subsection{Chọn dữ liệu để huấn luyện mô hình}
Trong quá trình nghiên cứu, việc lựa chọn tập dữ liệu phù hợp đóng vai trò quyết định đến hiệu suất và độ tin cậy của mô hình học máy. Sau khi phân tích cấu trúc và đặc điểm của bộ dữ liệu, chúng tôi quyết định tập trung vào các điểm đo có đầy đủ ba kênh đo (\texttt{chNumber} = 0, 1, 2) với các lý do sau:

\subsubsection{Tính đầy đủ của thông tin vật lý}
Các điểm đo có đủ ba kênh cung cấp bức tranh toàn diện về trạng thái thủy lực tại mỗi vị trí. Cụ thể, việc đồng thời có thông tin về áp suất đầu vào, lưu lượng, và áp suất đầu ra cho phép:

\begin{itemize}
    \item Tính toán chênh lệch áp suất ($\Delta P = P_{in} - P_{out}$) trực tiếp, là chỉ số quan trọng để phát hiện tổn thất áp suất bất thường.
    \item Thiết lập mối tương quan giữa lưu lượng và chênh lệch áp suất, tuân theo định luật Darcy-Weisbach và các nguyên lý thủy lực cơ bản.
    \item Xây dựng các đặc trưng phái sinh (derived features) có giá trị cao trong việc phát hiện rò rỉ, như hệ số tổn thất cục bộ, chỉ số biến thiên áp suất theo lưu lượng, v.v.
\end{itemize}

\subsubsection{Khả năng phát hiện bất thường đa chiều}
Với ba kênh đo, mô hình có thể phát hiện bất thường theo nhiều chiều và dạng biểu hiện khác nhau:

\begin{itemize}
    \item \textbf{Bất thường về áp suất đầu vào:} Có thể phản ánh vấn đề từ nguồn cung cấp hoặc đoạn ống phía trước.
    \item \textbf{Bất thường về lưu lượng:} Phát hiện tiêu thụ bất thường, rò rỉ lớn, hoặc sự cố van.
    \item \textbf{Bất thường về áp suất đầu ra:} Phản ánh vấn đề ở phía hạ nguồn hoặc tình trạng tắc nghẽn.
    \item \textbf{Bất thường về mối tương quan:} Quan trọng nhất, có thể phát hiện rò rỉ nhỏ thông qua sự thay đổi trong mối quan hệ giữa ba thông số, ngay cả khi mỗi thông số riêng lẻ vẫn nằm trong ngưỡng bình thường.
\end{itemize}

\subsubsection{Tính ổn định và độ tin cậy của mô hình}
Mô hình được huấn luyện trên dữ liệu đầy đủ ba kênh sẽ có độ tin cậy cao hơn vì:

\begin{itemize}
    \item Giảm thiểu sự phụ thuộc vào một kênh đo duy nhất, từ đó giảm tác động của nhiễu cục bộ hoặc lỗi cảm biến.
    \item Cho phép áp dụng các kỹ thuật kiểm tra chéo (cross-validation) giữa các kênh để xác nhận tính hợp lệ của dữ liệu.
    \item Tạo điều kiện cho việc áp dụng các phương pháp học sâu phức tạp hơn như mạng nơ-ron tích chập (CNN) hoặc mạng LSTM đa đầu vào, vốn hoạt động hiệu quả với dữ liệu đa chiều.
\end{itemize}

\subsubsection{Khả năng mở rộng và ứng dụng thực tế}
Mặc dù việc tập trung vào các điểm đo có đủ ba kênh có thể làm giảm số lượng điểm đo khả dụng trong tập dữ liệu, nhưng lợi ích về mặt chất lượng và khả năng ứng dụng thực tế là đáng kể:

\begin{itemize}
    \item Các mô hình được huấn luyện trên dữ liệu đầy đủ có thể được điều chỉnh để hoạt động với các điểm đo có ít kênh hơn thông qua các kỹ thuật như transfer learning hoặc feature imputation.
    \item Trong thực tế vận hành, các điểm đo quan trọng (như trạm bơm, điểm phân phối chính, hoặc khu vực có nguy cơ rò rỉ cao) thường được trang bị đầy đủ cảm biến, phù hợp với mô hình của chúng tôi.
    \item Kết quả nghiên cứu có thể được sử dụng để đề xuất cải tiến cấu hình đo lường trong tương lai, hướng tới việc nâng cấp các điểm đo hiện có lên cấu hình ba kênh.
\end{itemize}

\subsubsection{Phương pháp lựa chọn cụ thể}
Dựa trên các lý do trên, chúng tôi áp dụng quy trình lựa chọn dữ liệu như sau:

\begin{enumerate}
    \item Quét toàn bộ tập dữ liệu để xác định các điểm đo có đủ ba kênh (\texttt{chNumber} = 0, 1, 2).
    \item Đối với mỗi điểm đo được chọn, kiểm tra tính đầy đủ của dữ liệu theo thời gian (tỷ lệ giá trị thiếu không vượt quá 10\%).
    \item Kiểm tra tính hợp lệ của dữ liệu thông qua các ràng buộc vật lý (ví dụ: áp suất đầu vào luôn lớn hơn hoặc bằng áp suất đầu ra trong điều kiện bình thường).
    \item Ưu tiên các điểm đo có lịch sử ghi nhận sự cố hoặc bảo trì, tạo điều kiện để đánh giá khả năng phát hiện bất thường của mô hình.
\end{enumerate}

Kết quả của quá trình lựa chọn này là một tập con gồm $N$ điểm đo đáp ứng đầy đủ các tiêu chí, sẽ được sử dụng làm cơ sở cho các bước tiền xử lý và huấn luyện mô hình tiếp theo.
\subsubsection{Phân tích các dữ liệu tại từng điểm đo}
Các điểm đo thiếu dữ liệu

Các điểm đo có đủ dữ liệu để xử lý


\subsection{Xử lý dữ liệu thiếu}
\subsection{Feature selection}
\subsection{Tạo chuỗi dữ liệu huấn luyện}
\subsection{Tạo tập dữ liệu huấn luyện và kiểm tra}

\section{Huấn luyện mô hình}
\subsection{Tổng quan}
\subsection{Huấn luyện mô hình}
\subsection{Đánh giá mô hình}

\section{Phát hiện bất thường}
\subsection{Tổng quan}
\subsection{Phát hiện bằng sai số}
\subsection{Phát hiện bằng các phương pháp học không giám sát}
\subsubsection{Isolation Forest}
\subsubsection{One-class SVM}
\subsubsection{LOF}
\subsubsection{One-class SVM}
\subsection{Phân tích kết quả}


\section{Kết luận}

