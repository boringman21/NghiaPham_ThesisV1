\begin{tabular}
{p{0.4\textwidth} p{0.55\textwidth}}
     \begin{center}
    \fontsize{11}{15}\selectfont ĐẠI HỌC QUỐC GIA TP.HCM\\
         \fontsize{11}{15}\selectfont\textbf{TRƯỜNG ĐẠI HỌC BÁCH KHOA}\\
         -----------------------------
     \end{center}& \begin{center}
         \fontsize{11}{15}\selectfont\textbf{CỘNG HOÀ XÃ HỘI CHỦ NGHĨA VIỆT NAM}\\
         \fontsize{11}{15}\selectfont\textbf{Độc lập - Tự do - Hạnh phúc}\\
         -----------------------------
     \end{center}  
\end{tabular}

\begin{center}
    \fontsize{16}{20}\selectfont
    \textbf{NHIỆM VỤ LUẬN VĂN THẠC SĨ}

\end{center}
\begingroup
\renewcommand{\arraystretch}{1.5}
\begin{table}[h]
    \centering
    \begin{tabular}{p{0.5\textwidth}p{0.4\textwidth}}
        Họ tên học viên: Phạm Tuấn Nghĩa&MSHV: 2170546  \\
        Ngày, tháng, năm sinh: 21/12/1998 &Nơi sinh: TP Hải Dương  \\
        Chuyên ngành: Khoa học máy tính &Mã số: 8480101
    \end{tabular}
\end{table}
\endgroup
\subsubsection*{I. TÊN ĐỀ TÀI:}
\onehalfspacing
Ứng dụng học máy trong phát hiện rò rỉ hệ thống cấp nước (Application of machine learning for leak detection in
water supply systems)

\subsubsection*{II. NHIỆM VỤ VÀ NỘI DUNG}

\begin{itemize}
    \item Tìm hiểu và phân tích tập dữ liệu LeakDB nhằm xác định các yếu tố và đặc trưng dữ liệu có thể phản ánh hiện tượng rò rỉ nước, làm cơ sở cho việc xây dựng mô hình phát hiện bất thường.
    \item Tìm hiểu, phân tích và đề xuất các phương pháp xử lý dữ liệu bị thiếu phù hợp với dữ liệu chuỗi thời gian thu thập từ các trạm đo áp suất và lưu lượng trong hệ thống cấp nước thực tế.
    \item Tiến hành thực nghiệm huấn luyện các mô hình nhằm dự đoán lưu lượng nước, qua đó phát hiện các điểm rò rỉ dựa trên sự sai lệch giữa giá trị dự đoán và giá trị thực tế.
    \item Áp dụng các phương pháp phát hiện bất thường phi giám sát để xác định các điểm dữ liệu bất thường trong chuỗi đo đạc, hỗ trợ cho việc phát hiện rò rỉ.
    \item Đánh giá hiệu năng của các mô hình và phương pháp phát hiện bất thường dựa trên các chỉ số đánh giá như sai số tuyệt đối trung bình (MAE), từ đó xác định phương pháp phù hợp nhất với dữ liệu thực tế.
\end{itemize}

\subsubsection*{III. NGÀY GIAO NHIỆM VỤ:} 
13/01/2025

\subsubsection*{IV. NGÀY HOÀN THÀNH NHIỆM VỤ:}
12/05/2024

\subsubsection*{V. CÁN BỘ HƯỚNG DẪN:}
TS. Lê Thanh Vân

\vspace{1em}
\begin{tabular}
{p{0.4\textwidth} p{0.5\textwidth}}
\vspace{1.25em}
     \begin{center}
         \textbf{CÁN BỘ HƯỚNG DẪN}
     \end{center}& \begin{center}
     Tp. HCM, ngày    tháng 05 năm 2025
     \\
         \textbf{HỘI ĐỒNG NGÀNH}
     \end{center}   \\
\end{tabular}
\vspace{5em}
\begin{center}
    \textbf{TRƯỞNG KHOA KHOA HỌC VÀ KỸ THUẬT MÁY TÍNH}
\end{center}
