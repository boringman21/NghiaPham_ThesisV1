\chapter*{Abstract}

In the context of increasing demand for clean water and the pressure to effectively manage urban infrastructure systems, monitoring and detecting leaks in water supply networks plays a pivotal role in resource conservation and service quality enhancement. Water leakage not only causes significant economic losses but also poses risks of water quality degradation, operational imbalance disruption, and impacts on sustainable urban development.

This thesis focuses on investigating the application of modern machine learning techniques for water leak detection based on the analysis of actual operational data from water supply systems, particularly flow data. The research emphasizes the development of predictive models for normal water flow conditions, thereby identifying potential anomalies that may be associated with leakage phenomena.

Deep learning models such as Long Short-Term Memory (LSTM) and Gated Recurrent Unit (GRU) networks have been implemented to learn and model water flow time series. Based on the prediction error between the model and actual measurements, anomalous signals are identified as the initial basis for leak detection.

Furthermore, the thesis integrates and evaluates prominent anomaly detection techniques including Isolation Forest, Local Outlier Factor (LOF), and One-Class SVM. These methods operate independently of the predictive model and are utilized to detect anomalous data patterns in feature space, thereby enhancing multi-dimensional and comprehensive leak identification capabilities.

Experimental results on real-world data demonstrate that the combination of deep learning models and traditional anomaly detection algorithms yields high effectiveness in identifying anomalous events related to leaks. The research not only proves the feasibility and effectiveness of machine learning in urban infrastructure but also opens up potential for developing automated and intelligent monitoring systems for future water resource management.