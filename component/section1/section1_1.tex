\chapter{Mở đầu}
\section{Thực trạng rò rỉ nước trong hệ thống cấp nước đô thị}

Trong bối cảnh toàn cầu hóa và đô thị hóa ngày càng nhanh, việc đáp ứng nhu cầu sử dụng nước sạch và ổn định trở thành một trong những mục tiêu cấp thiết nhất đối với các đô thị trên toàn thế giới. Hệ thống cấp nước đô thị không chỉ đóng vai trò cung cấp nước sinh hoạt cho hàng triệu người mà còn đảm bảo duy trì các hoạt động công nghiệp, nông nghiệp và các dịch vụ thiết yếu khác. Tuy nhiên, cùng với sự phát triển của xã hội và hạ tầng đô thị, các vấn đề liên quan đến thất thoát nước, đặc biệt là rò rỉ trong hệ thống đường ống cấp nước, đã và đang trở thành thách thức lớn đối với nhiều quốc gia.

Hệ thống cấp nước là một mạng lưới phức tạp gồm các thành phần như đường ống, trạm bơm, bồn chứa, van điều áp và các thiết bị phụ trợ khác. Mạng lưới này thường kéo dài hàng ngàn kilômét trong các đô thị lớn và vận hành dưới mặt đất, nơi việc tiếp cận và bảo trì gặp nhiều khó khăn. Khi xảy ra rò rỉ, nước sạch bị thất thoát trước khi đến tay người tiêu dùng, gây ra nhiều hậu quả nghiêm trọng về kinh tế, môi trường và xã hội. 

\subsection{Vấn đề rò rỉ nước trong hệ thống cấp nước đô thị}

Rò rỉ nước là một hiện tượng phổ biến trong hệ thống cấp nước và có thể xảy ra ở bất kỳ giai đoạn nào của quá trình vận hành. Tình trạng này thường bắt nguồn từ nhiều nguyên nhân khác nhau, bao gồm sự xuống cấp của vật liệu ống dẫn (như thép, gang, nhựa PVC), sự ăn mòn hóa học và điện hóa, áp suất thủy lực không ổn định, chấn động do phương tiện giao thông hoặc động đất, cũng như các lỗi kỹ thuật trong quá trình lắp đặt và bảo trì. Ngoài ra, các tác động từ bên ngoài, chẳng hạn như sụt lún địa tầng hoặc thiên tai, cũng làm gia tăng nguy cơ rò rỉ nước.

Trong các khu vực đô thị, mật độ xây dựng cao và hệ thống ngầm phức tạp khiến việc phát hiện và sửa chữa rò rỉ trở nên khó khăn. Nhiều khu vực sử dụng cơ sở hạ tầng cấp nước đã cũ, vượt quá tuổi thọ thiết kế, làm tăng nguy cơ xảy ra sự cố. Điều này đặc biệt đúng đối với các thành phố lớn như Hà Nội và TP. Hồ Chí Minh, nơi có hệ thống cấp nước ngầm rộng lớn nhưng chưa được bảo trì đúng mức.

\subsection{Thống kê thực trạng}
Theo báo cáo của Hiệp hội Nước Mỹ (\textit{American Water Works Association - AWWA}), tỷ lệ thất thoát nước ở các hệ thống cấp nước trên toàn thế giới dao động từ 20\% đến 30\%~\cite{AWWA_WaterLossControl}. Ở các quốc gia đang phát triển, tỷ lệ này có thể vượt quá 40\%. Điều này có nghĩa là một phần ba hoặc thậm chí gần một nửa lượng nước sạch đã được xử lý với chi phí cao bị lãng phí. Tại Việt Nam, các công ty cấp nước đô thị như SAWACO\cite{SAWACO_WaterLoss} và Công ty Cấp nước Hà Nội\cite{HN_WaterLoss} báo cáo tỷ lệ thất thoát nước trung bình từ 15\% đến 30\%, tương đương với hàng triệu mét khối nước bị thất thoát mỗi năm.

\subsection{Hậu quả của rò rỉ nước}
Hậu quả của rò rỉ nước không chỉ giới hạn ở việc lãng phí tài nguyên mà còn ảnh hưởng sâu rộng đến kinh tế, môi trường và xã hội:
\begin{itemize}
    \item \textbf{Lãng phí tài nguyên nước:} Nước ngọt là nguồn tài nguyên quý giá, đặc biệt trong bối cảnh biến đổi khí hậu và gia tăng dân số. Việc thất thoát nước làm giảm hiệu quả khai thác và sử dụng nguồn tài nguyên này.
    \item \textbf{Thiệt hại kinh tế:} Các đơn vị cung cấp nước phải chịu thiệt hại về doanh thu, đồng thời phải tăng chi phí vận hành và sửa chữa. Thiệt hại này cuối cùng sẽ được chuyển sang người tiêu dùng dưới hình thức tăng giá nước.
    \item \textbf{Ảnh hưởng môi trường:} Lượng nước thất thoát có thể gây xói mòn đất, sụt lún hoặc ô nhiễm nguồn nước ngầm. Những vấn đề này gây ra tác động tiêu cực đến môi trường sống và cơ sở hạ tầng.
    \item \textbf{Suy giảm chất lượng dịch vụ:} Khi áp lực nước không ổn định do rò rỉ, việc cung cấp nước đến người tiêu dùng bị gián đoạn. Điều này làm giảm lòng tin của người dân vào dịch vụ cấp nước.
\end{itemize}

Tình trạng này cho thấy tầm quan trọng của việc áp dụng các giải pháp phát hiện và ngăn chặn rò rỉ nước hiệu quả để giảm thiểu những thiệt hại trên.
