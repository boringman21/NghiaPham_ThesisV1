\section{Tiền xử lý dữ liệu}
\subsection{Tổng quan}
...
% TODO: thêm lược đồ tại đây
\subsection{Chuyển đổi dữ liệu}
Quá trình chuyển đổi dữ liệu thô thành định dạng phù hợp cho phân tích và huấn luyện mô hình đóng vai trò then chốt trong tiền xử lý. Dữ liệu thô ban đầu từ hệ thống \texttt{find\_query\_x} cần được tái cấu trúc theo một khuôn mẫu chuẩn hóa nhằm tối ưu hóa cho các bước phân tích tiếp theo. Chúng tôi đã thiết lập và triển khai một quy trình chuyển đổi có hệ thống, được mô tả chi tiết như sau:

\subsubsection{Lọc dữ liệu theo mã thiết bị}
Bước đầu tiên trong quy trình chuyển đổi là phân tách dữ liệu theo mã định danh thiết bị (\texttt{smsNumber}). Mỗi thiết bị trong hệ thống có một mã định danh riêng biệt. Việc phân tách này mang lại các lợi ích sau:

\begin{itemize}
    \item Tách biệt dữ liệu từ các vị trí địa lý khác nhau trong mạng lưới
    \item Đảm bảo tính nhất quán của dữ liệu trong một điểm đo cụ thể
    \item Cho phép xây dựng các mô hình phù hợp với đặc điểm của từng điểm đo
    \item Tạo điều kiện thuận lợi cho việc so sánh giữa các điểm đo khác nhau
\end{itemize}

Quá trình lọc được thực hiện bằng cách truy vấn dữ liệu theo điều kiện mã thiết bị. Kết quả là một tập dữ liệu chỉ chứa thông tin từ một thiết bị cụ thể, sẵn sàng cho các bước xử lý tiếp theo.

% để hình CSV ở  đây
% \begin{figure}[h]
%     \centering
%     \includegraphics[width=0.8\textwidth]{image/section6_3/device_filtering.png}
%     \caption{Minh họa quá trình lọc dữ liệu theo mã thiết bị}
%     \label{fig:device_filtering}
% \end{figure}

\subsubsection{Tái cấu trúc dữ liệu theo kênh đo}
Sau khi hoàn tất việc lọc theo thiết bị, dữ liệu thô vẫn tồn tại ở dạng "dài" (long format), trong đó mỗi kênh đo được biểu diễn bằng nhiều bản ghi riêng biệt theo thời gian. Cấu trúc này, mặc dù hiệu quả cho việc lưu trữ, lại không tối ưu cho các phân tích đa biến và huấn luyện mô hình. Do đó, chúng tôi tiến hành chuyển đổi dữ liệu sang định dạng "rộng" (wide format) thông qua một quy trình có hệ thống:

\begin{enumerate}
    \item Nhóm dữ liệu theo các mốc thời gian (\texttt{Timestamp}) để tạo ra cấu trúc thời gian chuẩn
    \item Xác định và trích xuất giá trị từ mỗi kênh đo (\texttt{chNumber}) tại mỗi mốc thời gian
    \item Chuyển đổi các giá trị này thành các cột riêng biệt, tạo ra một cấu trúc ma trận trong đó mỗi hàng đại diện cho một mốc thời gian và mỗi cột đại diện cho một kênh đo
    \item Áp dụng các phép biến đổi bổ sung để đảm bảo tính nhất quán và đầy đủ của dữ liệu
\end{enumerate}

Trong quá trình này, chúng tôi áp dụng một hệ thống ánh xạ chuẩn hóa cho các kênh đo, nhằm tăng tính trực quan và dễ hiểu của dữ liệu:

\begin{itemize}
    \item Kênh 0 (\texttt{chNumber = 0}): Đại diện cho áp suất đầu vào, được gán nhãn là \texttt{Pressure\_1} trong tập dữ liệu đã chuyển đổi
    \item Kênh 1 (\texttt{chNumber = 1}): Đại diện cho lưu lượng, được gán nhãn là \texttt{Flow} trong tập dữ liệu đã chuyển đổi
    \item Kênh 2 (\texttt{chNumber = 2}): Đại diện cho áp suất đầu ra, được gán nhãn là \texttt{Pressure\_2} trong tập dữ liệu đã chuyển đổi
\end{itemize}

Việc ánh xạ này không chỉ tạo ra một cấu trúc dữ liệu rõ ràng mà còn phản ánh chính xác mối quan hệ vật lý giữa các thông số đo lường, tạo điều kiện thuận lợi cho các phân tích thủy lực và phát hiện bất thường trong hệ thống.

\subsubsection{Chuẩn hóa định dạng thời gian}
Việc chuẩn hóa thông tin thời gian đóng vai trò then chốt trong xử lý dữ liệu chuỗi thời gian. Chúng tôi áp dụng định dạng ISO 8601 (YYYY-MM-DD HH:MM:SS) cho cột \texttt{Timestamp}, đảm bảo tính nhất quán và chính xác trong các trường hợp:

\begin{itemize}
    \item Thực hiện các phép tính thời gian phức tạp như xác định khoảng cách giữa các mẫu hoặc phát hiện mẫu bị thiếu
    \item Sắp xếp dữ liệu theo trình tự thời gian chính xác để phân tích xu hướng
\end{itemize}

Quá trình chuẩn hóa bao gồm việc chuyển đổi tất cả biểu diễn thời gian sang định dạng datetime chuẩn và sắp xếp dữ liệu theo thứ tự thời gian tăng dần, tạo nền tảng vững chắc cho các phân tích chuỗi thời gian phức tạp trong các bước tiếp theo.

\subsubsection{Xử lý đơn vị đo}
Để đảm bảo tính nhất quán và khả năng so sánh giữa các điểm đo, chúng tôi thực hiện chuẩn hóa đơn vị đo cho tất cả các giá trị trong tập dữ liệu. Cụ thể, chúng tôi áp dụng các đơn vị chuẩn sau:

\begin{itemize}
    \item \textbf{Áp suất (\texttt{Pressure\_1} và \texttt{Pressure\_2}):} Đơn vị bar, một đơn vị áp suất phổ biến trong ngành cấp thoát nước, tương đương với khoảng 100.000 Pascal
    \item \textbf{Lưu lượng (\texttt{Flow}):} Đơn vị m³/h (mét khối trên giờ), đơn vị tiêu chuẩn để đo lường thể tích chất lỏng chảy qua một điểm trong một đơn vị thời gian
\end{itemize}

Trong trường hợp dữ liệu gốc sử dụng các đơn vị khác (ví dụ: psi cho áp suất hoặc lít/phút cho lưu lượng), chúng tôi áp dụng các hệ số chuyển đổi phù hợp để đảm bảo tính nhất quán trong toàn bộ tập dữ liệu. Việc chuẩn hóa đơn vị đo không chỉ tạo điều kiện thuận lợi cho việc phân tích và so sánh mà còn đảm bảo tính chính xác của các mô hình dự đoán và phát hiện bất thường.

\subsubsection{Kết quả chuyển đổi}
Sau khi hoàn tất quá trình chuyển đổi đa bước nêu trên, chúng tôi thu được một cấu trúc dữ liệu có tổ chức cao, được biểu diễn dưới dạng DataFrame với cấu trúc như sau:

\begin{table}[htbp]
    \centering
    \begin{tabular}{|c|c|c|c|}
        \hline
        \textbf{Timestamp} & \textbf{Pressure\_1} & \textbf{Flow} & \textbf{Pressure\_2} \\
        \hline
        2024-01-01 00:00:00 & 5.2 & 120.5 & 4.8 \\
        2024-01-01 00:15:00 & 5.3 & 118.7 & 4.9 \\
        2024-01-01 00:30:00 & 5.1 & 121.2 & 4.7 \\
        \ldots & \ldots & \ldots & \ldots \\
        \hline
    \end{tabular}
    \caption{Cấu trúc DataFrame sau khi chuyển đổi cho một điểm đo}
    \label{tab:transformed_data_structure}
\end{table}

DataFrame đã chuyển đổi này có các đặc điểm quan trọng sau:

\begin{itemize}
    \item \textbf{Cấu trúc thời gian nhất quán:} Các mẫu cách đều nhau 15 phút, tạo chuỗi thời gian đồng nhất
    \item \textbf{Tổ chức hợp lý:} Ba cột dữ liệu phản ánh đúng ý nghĩa vật lý của các thông số đo
    \item \textbf{Sắp xếp theo thời gian:} Dữ liệu được sắp xếp tăng dần theo thời gian, thuận lợi cho phân tích xu hướng
    \item \textbf{Tần suất lấy mẫu chuẩn:} Khoảng cách 15 phút giữa các mẫu cho phép phân tích chi tiết biến động trong ngày
\end{itemize}

Cấu trúc này tạo nền tảng vững chắc cho việc phân tích và huấn luyện mô hình.

\subsection{Chọn dữ liệu để huấn luyện mô hình}
Trong quá trình nghiên cứu, việc lựa chọn tập dữ liệu phù hợp đóng vai trò quyết định đến hiệu suất và độ tin cậy của mô hình học máy. Sau khi phân tích cấu trúc và đặc điểm của bộ dữ liệu, chúng tôi quyết định tập trung vào các điểm đo có đầy đủ ba kênh đo (\texttt{chNumber} = 0, 1, 2) với các lý do sau:

\subsubsection{Tính đầy đủ của thông tin vật lý}
Các điểm đo có đủ ba kênh cung cấp bức tranh toàn diện về trạng thái thủy lực tại mỗi vị trí. Cụ thể, việc đồng thời có thông tin về áp suất đầu vào, lưu lượng, và áp suất đầu ra cho phép:

\begin{itemize}
    \item Tính toán chênh lệch áp suất ($\Delta P = P_{in} - P_{out}$) trực tiếp, là chỉ số quan trọng để phát hiện tổn thất áp suất bất thường.
    \item Thiết lập mối tương quan giữa lưu lượng và chênh lệch áp suất, tuân theo định luật Darcy-Weisbach và các nguyên lý thủy lực cơ bản.
    \item Xây dựng các đặc trưng phái sinh (derived features) có giá trị cao trong việc phát hiện rò rỉ, như hệ số tổn thất cục bộ, chỉ số biến thiên áp suất theo lưu lượng, v.v.
\end{itemize}

\subsubsection{Khả năng phát hiện bất thường đa chiều}
Với ba kênh đo, mô hình có thể phát hiện bất thường theo nhiều chiều và dạng biểu hiện khác nhau:

\begin{itemize}
    \item \textbf{Bất thường về áp suất đầu vào:} Có thể phản ánh vấn đề từ nguồn cung cấp hoặc đoạn ống phía trước.
    \item \textbf{Bất thường về lưu lượng:} Phát hiện tiêu thụ bất thường, rò rỉ lớn, hoặc sự cố van.
    \item \textbf{Bất thường về áp suất đầu ra:} Phản ánh vấn đề ở phía hạ nguồn hoặc tình trạng tắc nghẽn.
    \item \textbf{Bất thường về mối tương quan:} Quan trọng nhất, có thể phát hiện rò rỉ nhỏ thông qua sự thay đổi trong mối quan hệ giữa ba thông số, ngay cả khi mỗi thông số riêng lẻ vẫn nằm trong ngưỡng bình thường.
\end{itemize}

\subsubsection{Tính ổn định và độ tin cậy của mô hình}
Mô hình được huấn luyện trên dữ liệu đầy đủ ba kênh sẽ có độ tin cậy cao hơn vì:

\begin{itemize}
    \item Giảm thiểu sự phụ thuộc vào một kênh đo duy nhất, từ đó giảm tác động của nhiễu cục bộ hoặc lỗi cảm biến.
    \item Cho phép áp dụng các kỹ thuật kiểm tra chéo (cross-validation) giữa các kênh để xác nhận tính hợp lệ của dữ liệu.
    \item Tạo điều kiện cho việc áp dụng các phương pháp học sâu phức tạp hơn như mạng nơ-ron tích chập (CNN) hoặc mạng LSTM đa đầu vào, vốn hoạt động hiệu quả với dữ liệu đa chiều.
\end{itemize}

\subsubsection{Phương pháp lựa chọn cụ thể}
Dựa trên các lý do trên, chúng tôi áp dụng quy trình lựa chọn dữ liệu như sau:

\begin{enumerate}
    \item Quét toàn bộ tập dữ liệu để xác định các điểm đo có đủ ba kênh (\texttt{chNumber} = 0, 1, 2).
    \item Đối với mỗi điểm đo được chọn, kiểm tra tính đầy đủ của dữ liệu theo thời gian.
\end{enumerate}

Kết quả của quá trình lựa chọn này là một tập con gồm $N$ điểm đo đáp ứng đầy đủ các tiêu chí, sẽ được sử dụng làm cơ sở cho các bước tiền xử lý và huấn luyện mô hình tiếp theo.

\subsubsection{Phân tích các dữ liệu tại từng điểm đo}
Sau khi tiến hành phân tích toàn diện, chúng tôi đã tổng hợp thông tin về tình trạng dữ liệu tại các điểm đo trong bảng dưới đây. Bảng \ref{tab:data-overview} trình bày tổng quan về các đặc điểm quan trọng của từng điểm đo, bao gồm mã định danh, khung thời gian đo, tình trạng dữ liệu lưu lượng và mức độ khuyết thiếu dữ liệu.

\begin{table}[htbp]
\centering
\resizebox{\textwidth}{!}{%
\begin{tabular}{|c|c|c|c|c|c|}
\hline
\textbf{smsNumber} & \textbf{Khung thời gian} & \textbf{Tình trạng lưu lượng} & \textbf{Mức độ khuyết thiếu} & \textbf{Số kênh đo} & \textbf{Ghi chú} \\
\hline
841210802047 & 15 phút & Không có dữ liệu & Rất cao (>69\%) & 3 kênh & Thiếu hoàn toàn dữ liệu lưu lượng \\
\hline
841210802048 & 15 phút & Không có dữ liệu & Thấp (3.84\%) & 3 kênh & 14 ngày dữ liệu bị mất, thiếu hoàn toàn dữ liệu lưu lượng \\
\hline
841210620665 & 15 phút & Không có dữ liệu & Thấp (3.01\%) & 3 kênh & 11 ngày dữ liệu bị mất, thiếu hoàn toàn dữ liệu lưu lượng \\
\hline
841210607378 & 15 phút & Không đầy đủ & Thấp (2.74\%) & 2 kênh & 10 ngày dữ liệu bị mất, không có dữ liệu flow \\
\hline
840786560116 & 15 phút & Không đầy đủ & Cao (45.21\%) & 3 kênh & Thiếu nhiều dữ liệu flow và áp suất đầu vào/ra \\
\hline
84797805118 & 15 phút & Không đầy đủ & Thấp (5.48\%) & 2 kênh & 20 ngày dữ liệu bị mất, không có áp suất trước, dữ liệu flow không đầy đủ \\
\hline
841212383325 & 5 phút & Không có dữ liệu & Rất cao (>51\%) & 2 kênh & Không có dữ liệu flow \\
\hline
841211914190 & 15 phút & Gần đầy đủ & Cao (32.05\%) & 3 kênh & 117 ngày dữ liệu bị mất \\
\hline
8401210607558 & 15 phút & Gần đầy đủ & Thấp (3.01\%) & 3 kênh & 11 ngày dữ liệu bị mất \\
\hline
\end{tabular}%
}
\caption{Tổng quan đặc điểm dữ liệu tại các điểm đo}
\label{tab:data-overview}
\end{table}

Từ bảng tổng quan này, có thể thấy rằng hầu hết các điểm đo đều có vấn đề về tính đầy đủ của dữ liệu, đặc biệt là dữ liệu lưu lượng. Nhiều điểm đo như 840786560116, 841210802048 và 841210620665 hoàn toàn thiếu dữ liệu lưu lượng, trong khi các điểm khác như 840786560116 và 84797805118 có dữ liệu lưu lượng không đầy đủ với tỷ lệ khuyết thiếu cao. Sau khi phân tích kỹ lưỡng, chúng tôi quyết định sử dụng hai điểm đo 841211914190 và 8401210607558 cho các thực nghiệm tiếp theo. Mặc dù điểm 841211914190 có tỷ lệ khuyết thiếu khá cao (32,05\%), nhưng dữ liệu lưu lượng gần đầy đủ và các giá trị khuyết thiếu xuất hiện liên tiếp nhau (117 ngày), thuận lợi cho việc áp dụng các phương pháp bổ khuyết. Tương tự, điểm 8401210607558 có tỷ lệ khuyết thiếu thấp (3,01\%) với dữ liệu lưu lượng gần đầy đủ và các giá trị khuyết thiếu cũng xuất hiện liên tiếp (11 ngày), phù hợp cho việc phân tích và mô hình hóa.

\subsubsection{Phân tích trực quan dữ liệu thiếu}

Để hiểu rõ hơn về đặc điểm và phân bố của dữ liệu thiếu, chúng tôi tiến hành phân tích trực quan đối với các điểm đo đã chọn. Hình \ref{fig:missing_data_visualization_841211914190} và Hình \ref{fig:missing_data_visualization_8401210607558} trình bày tổng quan về dữ liệu và các khoảng thời gian thiếu dữ liệu tại hai điểm đo có dữ liệu lưu lượng gần đầy đủ.

\begin{figure}[htbp]
    \centering
    \includegraphics[width=\textwidth]{image/section6_1/timeseries_combined_841211914190.png}
    \caption{Phân tích trực quan dữ liệu tại điểm đo 841211914190. Từ trên xuống dưới: Lưu lượng, Áp suất đầu vào, Áp suất đầu ra, Chênh lệch áp suất, và Biểu đồ dữ liệu thiếu theo thời gian.}
    \label{fig:missing_data_visualization_841211914190}
\end{figure}

\begin{figure}[htbp]
    \centering
    \includegraphics[width=\textwidth]{image/section6_1/timeseries_combined_8401210607558.png}
    \caption{Phân tích trực quan dữ liệu tại điểm đo 8401210607558. Từ trên xuống dưới: Lưu lượng, Áp suất đầu vào, Áp suất đầu ra, Chênh lệch áp suất, và Biểu đồ dữ liệu thiếu theo thời gian.}
    \label{fig:missing_data_visualization_8401210607558}
\end{figure}

Ngoài ra, chúng tôi cũng phân tích trực quan các điểm đo không có dữ liệu lưu lượng để hiểu rõ hơn về tình trạng dữ liệu trong toàn bộ hệ thống. Hình \ref{fig:no_flow_data_841210802047} và Hình \ref{fig:no_flow_data_841210620665} minh họa dữ liệu tại hai điểm đo thiếu hoàn toàn dữ liệu lưu lượng.

\begin{figure}[htbp]
    \centering
    \includegraphics[width=\textwidth]{image/section6_1/timeseries_combined_840786560116.png}
    \caption{Phân tích trực quan dữ liệu tại điểm đo 840786560116. Chỉ có dữ liệu áp suất đầu vào và đầu ra, hoàn toàn thiếu dữ liệu lưu lượng.}
    \label{fig:no_flow_data_841210802047}
\end{figure}

\begin{figure}[htbp]
    \centering
    \includegraphics[width=\textwidth]{image/section6_1/timeseries_combined_841210620665.png}
    \caption{Phân tích trực quan dữ liệu tại điểm đo 841210620665. Chỉ có dữ liệu áp suất đầu vào và đầu ra, hoàn toàn thiếu dữ liệu lưu lượng.}
    \label{fig:no_flow_data_841210620665}
\end{figure}

Từ các biểu đồ trực quan, có thể thấy rõ rằng dữ liệu thiếu tại điểm đo 841211914190 tập trung chủ yếu vào một khoảng thời gian liên tục kéo dài 117 ngày, trong khi tại điểm đo 8401210607558, dữ liệu thiếu xuất hiện trong một khoảng thời gian ngắn hơn (11 ngày). Đối với các điểm đo 841210802047 và 841210620665, mặc dù có dữ liệu áp suất tương đối đầy đủ, nhưng hoàn toàn thiếu dữ liệu lưu lượng, khiến chúng không phù hợp cho các phân tích cần thông tin về lưu lượng nước. Bảng \ref{tab:missing_data_841211914190} và Bảng \ref{tab:missing_data_8401210607558} cung cấp thông tin chi tiết về các ngày thiếu dữ liệu và số lượng điểm dữ liệu bị thiếu tại mỗi điểm đo.

\begin{table}[htbp]
\centering
\resizebox{\textwidth}{!}{%
\begin{tabular}{|c|c|c|}
\hline
\textbf{Khoảng thời gian} & \textbf{Số ngày} & \textbf{Số điểm dữ liệu thiếu/ngày} \\
\hline
2024-01-01 đến 2024-04-11 & 102 & 96 (toàn bộ ngày) \\
\hline
2024-04-12 & 1 & 53 \\
\hline
2024-09-10 đến 2024-09-23 & 14 & 13-75 (thiếu một phần) \\
\hline
\end{tabular}%
}
\caption{Tóm tắt dữ liệu thiếu tại điểm đo 841211914190}
\label{tab:missing_data_841211914190}
\end{table}

\begin{table}[htbp]
\centering
\begin{tabular}{|c|c|}
\hline
\textbf{Ngày} & \textbf{Số lượng điểm dữ liệu thiếu} \\
\hline
2024-02-29 & 2 \\
\hline
2024-09-10 & 84 \\
\hline
2024-09-11 & 28 \\
\hline
2024-09-12 & 36 \\
\hline
2024-09-13 & 46 \\
\hline
2024-09-14 & 30 \\
\hline
2024-09-15 & 34 \\
\hline
2024-09-16 & 26 \\
\hline
2024-09-17 & 40 \\
\hline
2024-09-18 & 24 \\
\hline
2024-09-19 & 32 \\
\hline
\end{tabular}
\caption{Chi tiết dữ liệu thiếu tại điểm đo 8401210607558}
\label{tab:missing_data_8401210607558}
\end{table}

Phân tích trực quan và thống kê chi tiết về dữ liệu thiếu giúp chúng tôi xác định rõ các khoảng thời gian cần được xử lý trong bước tiếp theo của quá trình tiền xử lý dữ liệu. Đặc biệt, việc hiểu rõ đặc điểm của dữ liệu thiếu sẽ giúp lựa chọn phương pháp bổ khuyết phù hợp nhất cho từng trường hợp cụ thể.

\subsection{Xử lý dữ liệu thiếu}
\subsubsection{Tổng quan phương pháp}
Dữ liệu thiếu là một thách thức phổ biến trong các chuỗi thời gian thực tế, có thể xuất hiện do nhiều nguyên nhân như sự cố thiết bị, mất kết nối truyền thông hoặc gián đoạn quá trình ghi nhận dữ liệu \cite{schafer2002missing}. Trong quá trình phân tích tại Phần \ref{sec:data-overview}, chúng tôi đã xác định rằng tập dữ liệu từ các điểm đo đều tồn tại hiện tượng dữ liệu thiếu ở những mức độ khác nhau, từ 2,74\% đến hơn 69\% tổng số điểm dữ liệu. Hiện tượng này đòi hỏi một chiến lược xử lý phù hợp để đảm bảo tính nhất quán và độ tin cậy của các mô hình dự đoán được huấn luyện trên dữ liệu.

Như đã trình bày trong Phần \ref{sec:model_evaluation}, các phương pháp xử lý dữ liệu thiếu trong chuỗi thời gian có thể được phân thành nhiều nhóm, bao gồm phương pháp nội suy, phương pháp dựa trên mô hình và phương pháp loại bỏ. Mỗi phương pháp đều có ưu nhược điểm riêng, và việc lựa chọn phương pháp phù hợp phụ thuộc vào nhiều yếu tố như tỷ lệ dữ liệu thiếu, cơ chế thiếu dữ liệu, và yêu cầu của bài toán.

\subsubsection{Phương pháp xóa ngày có dữ liệu thiếu (Listwise Deletion)}
Trong nghiên cứu này, chúng tôi lựa chọn áp dụng phương pháp loại bỏ toàn bộ các ngày có chứa bất kỳ điểm dữ liệu thiếu, còn được gọi là \textit{listwise deletion}. Phương pháp này đã được chứng minh là hiệu quả trong nhiều nghiên cứu về phát hiện bất thường trong chuỗi thời gian \cite{malhotra2016lstm, hundman2018detecting}.

Quyết định này dựa trên các cơ sở lý thuyết và thực tiễn sau:

\begin{enumerate}
    \item \textbf{Đảm bảo tính toàn vẹn của chuỗi thời gian}: Các mô hình học sâu như LSTM và GRU đòi hỏi tính liên tục của dữ liệu để duy trì trạng thái ẩn chính xác. Việc loại bỏ hoàn toàn các ngày có dữ liệu thiếu giúp đảm bảo rằng các chuỗi thời gian được sử dụng trong quá trình huấn luyện đều hoàn chỉnh và liên tục.
    
    \item \textbf{Giảm thiểu bias trong quá trình học}: Các phương pháp bổ khuyết dữ liệu như nội suy tuyến tính hoặc dự đoán từ mô hình có thể làm sai lệch phân phối xác suất gốc của dữ liệu, đặc biệt trong các bài toán nhạy cảm như phát hiện bất thường. Phương pháp loại bỏ giúp tránh việc đưa vào mô hình các giá trị không chính xác.
    
    \item \textbf{Phù hợp với đặc điểm chuỗi thời gian}: Dữ liệu chuỗi thời gian trong hệ thống giám sát thường thể hiện tính mùa vụ rõ rệt và sự biến động theo thời gian trong ngày. Việc loại bỏ toàn bộ các ngày có dữ liệu thiếu giúp bảo toàn các đặc điểm này trong các chuỗi thời gian còn lại.
\end{enumerate}

\subsubsection{Quy trình thực hiện}
Quá trình xử lý dữ liệu thiếu được thực hiện theo các bước sau:

\begin{enumerate}
    \item \textbf{Xác định các ngày có dữ liệu thiếu}: Đối với mỗi điểm đo, chúng tôi tiến hành phân tích chuỗi thời gian để xác định các ngày có chứa ít nhất một điểm dữ liệu thiếu trong bất kỳ kênh đo nào (áp suất đầu vào, lưu lượng, áp suất đầu ra). Một ngày được định nghĩa là khoảng thời gian từ 00:00 đến 23:59, bao gồm 96 điểm dữ liệu với tần suất lấy mẫu 15 phút.
    
    \item \textbf{Loại bỏ toàn bộ dữ liệu trong các ngày bị ảnh hưởng}: Thay vì chỉ loại bỏ các điểm dữ liệu thiếu cụ thể, chúng tôi loại bỏ toàn bộ dữ liệu của các ngày có chứa bất kỳ điểm dữ liệu thiếu nào. Điều này đảm bảo rằng các mẫu dữ liệu ngày được sử dụng trong các bước tiếp theo đều đầy đủ và nhất quán.
    
    \item \textbf{Chuẩn hóa chỉ số thời gian}: Sau khi loại bỏ các ngày có dữ liệu thiếu, chúng tôi cập nhật chỉ số thời gian để đảm bảo tính liên tục của chuỗi, giúp các mô hình học máy có thể xử lý dữ liệu một cách hiệu quả.
\end{enumerate}

\subsubsection{Kết quả xử lý dữ liệu thiếu}
Việc áp dụng phương pháp loại bỏ ngày có dữ liệu thiếu đã mang lại những kết quả đáng chú ý:

\begin{itemize}
    \item \textbf{Đối với điểm đo 841211914190}: Sau khi loại bỏ các ngày có dữ liệu thiếu, tập dữ liệu còn lại bao gồm 248 ngày đầy đủ, chiếm khoảng 67,95\% tổng số ngày trong khoảng thời gian thu thập dữ liệu. Các ngày bị loại bỏ chủ yếu tập trung vào khoảng thời gian từ 2024-01-01 đến 2024-04-11 (102 ngày) và một số ngày rải rác trong tháng 9/2024.
    
    \item \textbf{Đối với điểm đo 8401210607558}: Tập dữ liệu sau khi xử lý bao gồm 354 ngày đầy đủ, chiếm 96,99\% tổng số ngày. Các ngày bị loại bỏ chủ yếu nằm trong khoảng thời gian từ 2024-09-10 đến 2024-09-19, cùng với một ngày riêng lẻ (2024-02-29).
\end{itemize}

Mặc dù phương pháp này dẫn đến việc giảm số lượng dữ liệu huấn luyện, đặc biệt đối với điểm đo 841211914190, nhưng lợi ích từ việc đảm bảo tính nhất quán và độ tin cậy của dữ liệu vượt trội hơn hạn chế về khối lượng dữ liệu. Hơn nữa, với đặc điểm tính mùa vụ rõ rệt trong dữ liệu chuỗi thời gian hệ thống cấp nước, các mẫu dữ liệu còn lại vẫn đủ đại diện cho các mẫu hành vi điển hình của hệ thống.

% \begin{figure}[htbp]
%     \centering
%     \includegraphics[width=\textwidth]{image/section6_3/time_series_after_cleaning.png}
%     \caption{Biểu đồ minh họa chuỗi thời gian sau khi áp dụng phương pháp xóa ngày có dữ liệu thiếu tại điểm đo 841211914190}
%     \label{fig:time_series_after_cleaning}
% \end{figure}

Thông qua phương pháp xử lý dữ liệu thiếu này, chúng tôi đã tạo ra các tập dữ liệu sạch và đầy đủ, tạo nền tảng vững chắc cho các bước tiếp theo trong quá trình tiền xử lý dữ liệu và huấn luyện mô hình học máy. Phương pháp này cũng phù hợp với quan điểm được đề xuất trong nghiên cứu của Schafer và Graham \cite{schafer2002missing}, rằng khi tỷ lệ dữ liệu thiếu tương đối thấp và cơ chế thiếu dữ liệu là ngẫu nhiên hoàn toàn (MCAR), việc loại bỏ các mẫu có dữ liệu thiếu là một chiến lược hợp lý và hiệu quả.

\subsection{Feature selection}

Quá trình lựa chọn đặc trưng (\textit{feature selection}) đóng vai trò then chốt trong xây dựng mô hình dự đoán hiệu quả. Đối với bài toán dự báo lưu lượng nước và phát hiện rò rỉ trong hệ thống cấp nước, việc xác định tập đặc trưng phù hợp không chỉ ảnh hưởng đến độ chính xác mà còn quyết định khả năng mô hình hóa mối quan hệ thủy lực phức tạp trong hệ thống.

\subsubsection{Cơ sở lý thuyết cho việc lựa chọn đặc trưng}

Như đã trình bày trong Phần \ref{sec:ml_methods}, các mô hình học máy cần được cung cấp đặc trưng đầu vào phù hợp để tối ưu hóa khả năng học và dự đoán. Dựa trên hiểu biết về tính mùa vụ và các đặc điểm của chuỗi thời gian được đề cập trong Phần \ref{sec:data-overview}, chúng tôi xác định rằng dữ liệu lưu lượng nước thể hiện các mẫu hành vi tự tương quan mạnh mẽ theo thời gian, đồng thời cũng chịu ảnh hưởng bởi các yếu tố thủy lực khác như áp suất đầu vào và đầu ra.

Việc lựa chọn đặc trưng trong nghiên cứu này được thực hiện dựa trên cả kiến thức chuyên ngành thủy lực và phân tích thực nghiệm, hướng đến việc khám phá hai chiến lược mô hình hóa khác nhau:

\subsubsection{Chiến lược 1: Mô hình đơn biến sử dụng chuỗi lưu lượng}

Chiến lược đầu tiên tập trung vào việc xây dựng mô hình dự báo lưu lượng chỉ dựa trên dữ liệu lưu lượng lịch sử. Phương pháp này dựa trên nguyên lý rằng chuỗi thời gian lưu lượng thường thể hiện tính tự tương quan (\textit{autocorrelation}) mạnh mẽ, trong đó các giá trị trong quá khứ chứa thông tin có giá trị để dự đoán giá trị tương lai.

Cụ thể, các đặc trưng đầu vào cho mô hình này bao gồm:
\begin{itemize}
    \item \textbf{Chuỗi lưu lượng trễ} (\textit{lagged flow series}): Đây là tập hợp các giá trị lưu lượng trong $n$ khoảng thời gian trước đó, tức là $\{Flow_{t-n}, Flow_{t-n+1}, \ldots, Flow_{t-1}\}$ để dự đoán $Flow_t$
\end{itemize}

Việc sử dụng các giá trị trễ này dựa trên nhận định rằng mẫu tiêu thụ nước thường lặp lại theo chu kỳ nhất định (24 giờ, 7 ngày), như đã được chứng minh trong nghiên cứu về tính mùa vụ của dữ liệu chuỗi thời gian. Bằng cách khai thác tính mùa vụ và các mẫu tự tương quan trong dữ liệu lưu lượng, mô hình đơn biến có thể nắm bắt được các mẫu tiêu thụ nước định kỳ và dự đoán giá trị lưu lượng tương lai với độ chính xác hợp lý trong điều kiện vận hành bình thường.

Ưu điểm chính của chiến lược này là:
\begin{itemize}
    \item Đơn giản và dễ triển khai
    \item Yêu cầu ít dữ liệu đầu vào, phù hợp khi thiếu thông tin từ các cảm biến khác
    \item Mô hình hóa hiệu quả tính mùa vụ và các mẫu lặp lại trong dữ liệu lưu lượng
\end{itemize}

Tuy nhiên, chiến lược này có hạn chế là khả năng phát hiện các bất thường không tuân theo mẫu lịch sử chỉ dựa vào dữ liệu lưu lượng có thể bị hạn chế, đặc biệt khi rò rỉ xảy ra chậm và từ từ.

\subsubsection{Chiến lược 2: Mô hình đa biến kết hợp lưu lượng và áp suất}

Chiến lược thứ hai mở rộng tập đặc trưng đầu vào bằng cách kết hợp cả dữ liệu lưu lượng và áp suất, dựa trên nguyên lý thủy lực cơ bản là mối quan hệ giữa áp suất và lưu lượng trong hệ thống ống dẫn. Như được mô tả trong phương trình Darcy-Weisbach, tổn thất áp suất trong ống tỷ lệ thuận với bình phương lưu lượng trong điều kiện dòng chảy rối. Do đó, sự thay đổi trong mối quan hệ này có thể là dấu hiệu của rò rỉ hoặc các bất thường khác trong hệ thống.

Các đặc trưng đầu vào cho mô hình đa biến bao gồm:
\begin{itemize}
    \item \textbf{Chuỗi lưu lượng trễ}: $\{Flow_{t-n}, Flow_{t-n+1}, \ldots, Flow_{t-1}\}$
    \item \textbf{Chuỗi áp suất đầu vào trễ}: $\{Pressure1_{t-n}, Pressure1_{t-n+1}, \ldots, Pressure1_{t-1}\}$
    \item \textbf{Chuỗi áp suất đầu ra trễ}: $\{Pressure2_{t-n}, Pressure2_{t-n+1}, \ldots, Pressure2_{t-1}\}$
\end{itemize}

Việc kết hợp cả ba chuỗi thời gian này cho phép mô hình nắm bắt không chỉ các mẫu lưu lượng mà còn cả mối tương quan phức tạp giữa áp suất và lưu lượng. Điều này đặc biệt có giá trị trong việc phát hiện rò rỉ, vì các sự cố rò rỉ thường biểu hiện thông qua sự thay đổi trong mối quan hệ áp suất-lưu lượng trước khi chúng trở nên rõ ràng trong dữ liệu lưu lượng đơn thuần.

Ưu điểm của chiến lược này bao gồm:
\begin{itemize}
    \item Khả năng mô hình hóa mối quan hệ thủy lực phức tạp
    \item Tăng cường nhạy cảm đối với các dạng rò rỉ khác nhau, bao gồm cả rò rỉ nhỏ và phát triển chậm
    \item Giảm thiểu tác động của nhiễu cục bộ trong dữ liệu từ một cảm biến đơn lẻ
\end{itemize}

Tuy nhiên, chiến lược này cũng đòi hỏi nhiều dữ liệu đầu vào hơn và có thể gặp thách thức khi một số cảm biến gặp sự cố hoặc cung cấp dữ liệu không đáng tin cậy.

\subsubsection{Cân nhắc về kích thước cửa sổ thời gian}

Đối với cả hai chiến lược, việc xác định kích thước cửa sổ thời gian $n$ (số lượng điểm dữ liệu trong quá khứ được sử dụng để dự đoán) là một cân nhắc quan trọng. Trong nghiên cứu này, chúng tôi xác định rằng:

\begin{itemize}
    \item Cửa sổ ngắn (ví dụ: 1-6 giờ, tương đương 4-24 điểm dữ liệu với tần suất lấy mẫu 15 phút) thích hợp cho việc nắm bắt các biến động ngắn hạn và phản ứng nhanh với các thay đổi đột ngột.
    \item Cửa sổ trung bình (ví dụ: 24 giờ, tương đương 96 điểm dữ liệu) cho phép mô hình học được mẫu tiêu thụ hàng ngày.
    \item Cửa sổ dài (ví dụ: 7 ngày, tương đương 672 điểm dữ liệu) giúp nắm bắt các mẫu tiêu thụ hàng tuần nhưng đòi hỏi nhiều tài nguyên tính toán hơn.
\end{itemize}

Trong thực tế vận hành hệ thống cấp nước, các chuyên gia thường sử dụng khung thời gian 7 ngày để theo dõi và phát hiện bất thường. Khung thời gian này cho phép họ quan sát đầy đủ các chu kỳ tiêu thụ nước theo ngày trong tuần, bao gồm cả sự khác biệt giữa ngày làm việc và cuối tuần. Tuy nhiên, trong thực nghiệm của chúng tôi, sau khi cân nhắc kỹ lưỡng, chúng tôi quyết định sử dụng cửa sổ 6 giờ (24 điểm dữ liệu) vì những lý do sau:

\begin{itemize}
    \item \textbf{Phản ứng nhanh với bất thường}: Cửa sổ ngắn hơn cho phép mô hình phát hiện và phản ứng nhanh hơn với các sự cố rò rỉ đột ngột, giảm thời gian phát hiện từ vài ngày xuống còn vài giờ.
    \item \textbf{Giảm độ trễ tính toán}: Với khối lượng dữ liệu nhỏ hơn, mô hình có thể được huấn luyện và cập nhật thường xuyên hơn, phù hợp với yêu cầu giám sát thời gian thực.
    \item \textbf{Tập trung vào biến động ngắn hạn}: Nhiều loại rò rỉ thể hiện dấu hiệu rõ ràng nhất trong các biến động ngắn hạn của lưu lượng và áp suất, mà cửa sổ dài có thể làm mờ đi.
    \item \textbf{Tối ưu hóa tài nguyên}: Giảm đáng kể yêu cầu về bộ nhớ và sức mạnh tính toán, cho phép triển khai trên các thiết bị cạnh với tài nguyên hạn chế.
\end{itemize}

Mặc dù cửa sổ 24 giờ hoặc 7 ngày có thể nắm bắt tốt hơn các mẫu tiêu thụ theo chu kỳ, thực nghiệm của chúng tôi cho thấy cửa sổ 6 giờ cung cấp sự cân bằng tối ưu giữa độ chính xác dự đoán và khả năng phát hiện bất thường kịp thời, đặc biệt trong bối cảnh phát hiện rò rỉ nước - nơi thời gian phản ứng là yếu tố then chốt để giảm thiểu thiệt hại.

\subsubsection{Xử lý mục tiêu dự đoán}

Cả hai chiến lược đều nhắm đến việc dự đoán lưu lượng tại thời điểm $t$ ($Flow_t$) dựa trên dữ liệu lịch sử. Tuy nhiên, trong một số trường hợp, việc dự đoán chênh lệch lưu lượng hoặc tỷ lệ thay đổi có thể mang lại hiệu quả tốt hơn cho việc phát hiện bất thường. Trong phạm vi nghiên cứu này, chúng tôi tập trung vào dự đoán trực tiếp giá trị lưu lượng tuyệt đối, vì phương pháp này đơn giản hơn và phù hợp với mục tiêu xây dựng mô hình cơ sở.

Việc lựa chọn đặc trưng như đã trình bày tạo nền tảng vững chắc cho các bước tiếp theo trong quá trình xây dựng mô hình dự đoán lưu lượng và phát hiện bất thường trong hệ thống cấp nước. Phương pháp tiếp cận kép này không chỉ cho phép so sánh hiệu suất của mô hình đơn biến và đa biến mà còn mang lại hiểu biết sâu sắc về mối quan hệ phức tạp giữa các thông số thủy lực trong hệ thống cấp nước.

\subsection{Tạo chuỗi dữ liệu huấn luyện}

Dựa trên chiến lược và cửa sổ thời gian đã được xác định ở phần trước, chúng tôi tiến hành tạo chuỗi dữ liệu huấn luyện cho các mô hình dự báo. Với tần suất lấy mẫu 15 phút của hệ thống SCADA, cửa sổ thời gian 6 giờ tương ứng với 24 điểm dữ liệu liên tiếp. Quá trình tạo chuỗi dữ liệu được thực hiện như sau:

\begin{itemize}
    \item \textbf{Dữ liệu đầu vào (X)}: Mỗi mẫu đầu vào bao gồm 24 điểm dữ liệu liên tiếp, tương ứng với chuỗi thời gian 6 giờ. Tùy theo chiến lược được áp dụng, dữ liệu đầu vào có thể là:
    \begin{itemize}
        \item \textbf{Chiến lược 1 (đơn biến)}: $X = \{Flow_{t-24}, Flow_{t-23}, \ldots, Flow_{t-1}\}$
        \item \textbf{Chiến lược 2 (đa biến)}: $X = \{Flow_{t-24}, \ldots, Flow_{t-1}, Pressure1_{t-24}, \ldots, Pressure1_{t-1}, Pressure2_{t-24}, \ldots, Pressure2_{t-1}\}$
    \end{itemize}
    
    \item \textbf{Dữ liệu đầu ra (y)}: Giá trị lưu lượng tại thời điểm tiếp theo $Flow_t$, tương ứng với dự báo cho 15 phút kế tiếp.
\end{itemize}

Chuỗi dữ liệu được tạo bằng phương pháp cửa sổ trượt (sliding window), trong đó:

\begin{enumerate}
    \item Cửa sổ đầu tiên bao gồm các điểm dữ liệu từ $t=1$ đến $t=24$ làm đầu vào, và điểm $t=25$ làm đầu ra.
    \item Cửa sổ tiếp theo trượt một bước (15 phút), bao gồm các điểm từ $t=2$ đến $t=25$ làm đầu vào, và điểm $t=26$ làm đầu ra.
    \item Quá trình này tiếp tục cho đến khi đạt đến cuối chuỗi dữ liệu.
\end{enumerate}

\begin{table}[h]
    \centering
    \begin{tabular}{|c|c|c|}
        \hline
        \textbf{Mẫu} & \textbf{Dữ liệu đầu vào (X)} & \textbf{Dữ liệu đầu ra (y)} \\
        \hline
        1 & $\{Flow_1, Flow_2, \ldots, Flow_{24}\}$ & $Flow_{25}$ \\
        \hline
        2 & $\{Flow_2, Flow_3, \ldots, Flow_{25}\}$ & $Flow_{26}$ \\
        \hline
        3 & $\{Flow_3, Flow_4, \ldots, Flow_{26}\}$ & $Flow_{27}$ \\
        \hline
        $\vdots$ & $\vdots$ & $\vdots$ \\
        \hline
        n & $\{Flow_{n}, Flow_{n+1}, \ldots, Flow_{n+23}\}$ & $Flow_{n+24}$ \\
        \hline
    \end{tabular}
    \caption{Minh họa phương pháp cửa sổ trượt trong tạo dữ liệu huấn luyện}
    \label{tab:sliding_window_example}
\end{table}

Phương pháp này tạo ra một tập dữ liệu huấn luyện với số lượng mẫu gần bằng tổng số điểm dữ liệu ban đầu trừ đi kích thước cửa sổ. Với dữ liệu một năm (35.040 điểm với tần suất 15 phút), chúng tôi thu được khoảng 35.016 mẫu huấn luyện.

\begin{table}[h]
    \centering
    \begin{tabular}{|p{3cm}|p{5cm}|p{5cm}|}
        \hline
        \textbf{Đặc điểm} & \textbf{Chiến lược 1 (Đơn biến)} & \textbf{Chiến lược 2 (Đa biến)} \\
        \hline
        Số lượng đặc trưng & 24 & 72 (24 × 3 chuỗi) \\
        \hline
        Loại dữ liệu & Chỉ lưu lượng & Lưu lượng và áp suất (đầu vào/đầu ra) \\
        \hline
        Kích thước ma trận đầu vào & n × 24 & n × 72 \\
        \hline
        Ưu điểm & Đơn giản, yêu cầu ít dữ liệu & Nắm bắt mối quan hệ thủy lực phức tạp \\
        \hline
        Nhược điểm & Không nắm bắt mối quan hệ áp suất-lưu lượng & Phức tạp hơn, yêu cầu nhiều dữ liệu \\
        \hline
    \end{tabular}
    \caption{So sánh cấu trúc dữ liệu đầu vào giữa hai chiến lược}
    \label{tab:strategy_input_comparison}
\end{table}

Việc chuẩn hóa dữ liệu cũng được thực hiện trong quá trình này để đảm bảo hiệu suất tối ưu của các mô hình học máy. Các giá trị đầu vào được chuẩn hóa về khoảng [0,1] hoặc có trung bình 0 và độ lệch chuẩn 1, tùy thuộc vào yêu cầu cụ thể của từng thuật toán.

Cách tiếp cận này cho phép mô hình học được các mẫu biến động ngắn hạn trong dữ liệu lưu lượng và áp suất, đồng thời duy trì khả năng phản ứng nhanh với các thay đổi đột ngột - yếu tố quan trọng trong việc phát hiện rò rỉ kịp thời.

\subsection{Tạo tập dữ liệu huấn luyện và kiểm tra}

Để đảm bảo tính khách quan trong đánh giá hiệu suất của mô hình, chúng tôi chia tập dữ liệu theo tỷ lệ 80:20 cho quá trình huấn luyện và kiểm tra. Cụ thể, 80\% dữ liệu đầu tiên được sử dụng làm tập huấn luyện (training set) và 20\% dữ liệu còn lại dùng làm tập kiểm tra (test set).

Chúng tôi áp dụng hai chiến lược khác nhau trong việc tạo dữ liệu huấn luyện:

\begin{enumerate}
    \item \textbf{Chiến lược 1 - Mô hình riêng cho từng điểm đo:} Dữ liệu được tách biệt theo từng điểm đo cụ thể. Mỗi điểm đo sẽ có tập dữ liệu huấn luyện và kiểm tra riêng, từ đó huấn luyện một mô hình dự báo chuyên biệt cho điểm đo đó. Cách tiếp cận này cho phép mô hình nắm bắt được đặc thù riêng của từng vị trí trong mạng lưới.
    
    \item \textbf{Chiến lược 2 - Mô hình chung cho tất cả điểm đo:} Dữ liệu từ tất cả các điểm đo được gộp lại thành một tập dữ liệu lớn duy nhất. Từ đó, một mô hình duy nhất được huấn luyện để dự báo cho tất cả các điểm đo trong hệ thống. Chiến lược này giúp mô hình học được các mối tương quan tổng thể và có khả năng tổng quát hóa tốt hơn cho toàn bộ mạng lưới.
\end{enumerate}

\begin{table}[h]
    \centering
    \begin{tabular}{|p{4cm}|p{5cm}|p{5cm}|}
        \hline
        \textbf{Đặc điểm} & \textbf{Chiến lược 1} & \textbf{Chiến lược 2} \\
        \hline
        Phạm vi mô hình & Mô hình riêng cho từng điểm đo & Mô hình chung cho tất cả điểm đo \\
        \hline
        Tập dữ liệu & Tách biệt theo từng điểm đo & Gộp từ tất cả điểm đo \\
        \hline
        Ưu điểm & Nắm bắt đặc thù riêng của từng vị trí, độ chính xác cục bộ cao & Học được mối tương quan tổng thể, tổng quát hóa tốt, tiết kiệm tài nguyên \\
        \hline
        Nhược điểm & Tốn nhiều tài nguyên tính toán, khó tổng quát hóa & Có thể bỏ qua đặc thù riêng của từng điểm đo \\
        \hline
        Phù hợp với & Hệ thống có điểm đo ít, đặc thù riêng biệt cao & Hệ thống lớn, nhiều điểm đo có tính chất tương đồng \\
        \hline
    \end{tabular}
    \caption{So sánh hai chiến lược tạo dữ liệu huấn luyện}
    \label{tab:training_strategy_comparison}
\end{table}

Mỗi chiến lược đều có ưu điểm riêng: chiến lược 1 tập trung vào độ chính xác cục bộ, trong khi chiến lược 2 hướng đến khả năng tổng quát hóa và hiệu quả về mặt tài nguyên tính toán. Việc so sánh hiệu suất của hai chiến lược này sẽ giúp xác định phương pháp tối ưu cho hệ thống phát hiện rò rỉ.


\section{Huấn luyện mô hình}
\subsection{Tổng quan}
% TODO: tạo lược đồ tại đây
\subsection{Huấn luyện mô hình}

Từ dữ liệu chuỗi thời gian đã được tiền xử lý ở các phần trước, chúng tôi tiến hành huấn luyện các mô hình học sâu cho bài toán dự báo lưu lượng nước. Trong quá trình này, hai chiến lược huấn luyện khác nhau được triển khai nhằm khám phá phương pháp tối ưu cho hệ thống dự báo và phát hiện bất thường.

\subsubsection{Các chiến lược huấn luyện}

\begin{itemize}
    \item \textbf{Chiến lược 1 - Mô hình riêng cho từng điểm đo:} Mỗi điểm đo trong hệ thống được huấn luyện một mô hình riêng biệt, với dữ liệu đầu vào chỉ từ điểm đo đó. Cách tiếp cận này cho phép mô hình nắm bắt các đặc thù cục bộ, như mẫu tiêu thụ nước đặc trưng, tính mùa vụ, và các đặc điểm thủy lực riêng tại mỗi vị trí trong mạng lưới cấp nước.
    
    \item \textbf{Chiến lược 2 - Mô hình chung cho tất cả điểm đo:} Một mô hình duy nhất được huấn luyện trên dữ liệu kết hợp từ tất cả các điểm đo. Chiến lược này nhằm tạo ra một mô hình tổng quát có khả năng dự báo trên nhiều điểm đo khác nhau, đồng thời học được các mẫu và mối tương quan chung trong toàn bộ hệ thống cấp nước.
\end{itemize}

Việc so sánh hiệu suất của hai chiến lược này không chỉ giúp xác định phương pháp huấn luyện tối ưu mà còn cung cấp hiểu biết sâu sắc về mức độ tương đồng hoặc khác biệt trong đặc điểm của các điểm đo.

\subsubsection{Kiến trúc mô hình học sâu}

Dựa trên khả năng xử lý dữ liệu chuỗi thời gian và tính hiệu quả đã được chứng minh trong nhiều nghiên cứu trước đây, chúng tôi chọn hai kiến trúc mạng nơ-ron hồi quy phổ biến là LSTM (Long Short-Term Memory) và GRU (Gated Recurrent Unit) cho bài toán dự báo lưu lượng.

\paragraph{Mô hình LSTM}
Mạng LSTM được thiết kế với khả năng nắm bắt phụ thuộc dài hạn trong chuỗi thời gian thông qua cơ chế cổng (gate mechanism), cho phép mô hình quyết định thông tin nào cần được lưu trữ, cập nhật hoặc loại bỏ trong quá trình xử lý chuỗi. Kiến trúc mô hình LSTM được triển khai như sau:

\begin{itemize}
    \item Lớp LSTM đầu tiên với 50 đơn vị, trả về chuỗi đầu ra đầy đủ
    \item Lớp Dropout với tỷ lệ 20\% để ngăn ngừa hiện tượng quá khớp
    \item Lớp LSTM thứ hai với 25 đơn vị (giảm một nửa so với lớp đầu tiên)
    \item Lớp Dropout thứ hai với tỷ lệ 20\%
    \item Lớp kết nối đầy đủ (Dense) với 1 đơn vị đầu ra, đại diện cho giá trị lưu lượng dự đoán
\end{itemize}

Mô hình được biên dịch với hàm tối ưu Adam, tốc độ học 0,001, hàm mất mát MSE (Mean Squared Error) và đo lường hiệu suất MAE (Mean Absolute Error).

\paragraph{Mô hình GRU}
Mạng GRU là một biến thể đơn giản hóa của LSTM, sử dụng ít tham số hơn nhưng vẫn duy trì hiệu quả trong việc nắm bắt phụ thuộc thời gian. Kiến trúc mô hình GRU được triển khai tương tự như LSTM:

\begin{itemize}
    \item Lớp GRU đầu tiên với 50 đơn vị, trả về chuỗi đầu ra đầy đủ
    \item Lớp Dropout với tỷ lệ 20\%
    \item Lớp GRU thứ hai với 25 đơn vị
    \item Lớp Dropout thứ hai với tỷ lệ 20\%
    \item Lớp kết nối đầy đủ với 1 đơn vị đầu ra
\end{itemize}

Mô hình GRU cũng được biên dịch với cấu hình tương tự như mô hình LSTM.

\subsubsection{Quá trình huấn luyện}

Quy trình huấn luyện đối với cả hai mô hình được thiết kế nhằm tối ưu hóa hiệu suất và khả năng tổng quát hóa. Các tham số huấn luyện chính bao gồm:

\begin{itemize}
    \item \textbf{Epochs:} Tối đa 100 chu kỳ huấn luyện, kết hợp với cơ chế dừng sớm
    \item \textbf{Batch size:} 32 mẫu, cân bằng giữa hiệu suất tính toán và tốc độ hội tụ
    \item \textbf{Early stopping:} Dừng huấn luyện nếu không có cải thiện về val\_loss sau 10 epoch, giúp tránh quá khớp
    \item \textbf{Model checkpoint:} Lưu mô hình có hiệu suất tốt nhất trên tập validation
\end{itemize}

Quá trình huấn luyện được hỗ trợ bởi các cơ chế giám sát hiệu suất trực quan, bao gồm biểu đồ theo dõi sự biến thiên của các độ đo mất mát (loss) và sai số tuyệt đối trung bình (MAE) trong quá trình huấn luyện.

\subsubsection{Triển khai các chiến lược huấn luyện}

Đối với mỗi chiến lược, chúng tôi tiến hành huấn luyện các mô hình như sau:

\paragraph{Chiến lược 1: Mô hình riêng cho từng điểm đo}
\begin{itemize}
    \item Cho mỗi điểm đo (841211914190 và 8401210607558), dữ liệu được chia thành tập huấn luyện (80\%) và tập kiểm tra (20\%).
    \item Hai mô hình LSTM và GRU được huấn luyện riêng biệt cho mỗi điểm đo với cùng cấu hình.
    \item Với hai loại dữ liệu đầu vào (chỉ lưu lượng và lưu lượng + áp suất), tổng cộng sẽ có 8 mô hình được huấn luyện (2 điểm đo × 2 kiến trúc × 2 loại dữ liệu đầu vào).
    \item Mỗi mô hình được lưu với định danh riêng biệt để dễ dàng theo dõi và đánh giá.
\end{itemize}

\paragraph{Chiến lược 2: Mô hình chung cho tất cả điểm đo}
\begin{itemize}
    \item Dữ liệu từ cả hai điểm đo được kết hợp lại thành một tập dữ liệu duy nhất.
    \item Tập dữ liệu kết hợp này cũng được chia theo tỷ lệ 80:20 cho quá trình huấn luyện và kiểm tra.
    \item Hai mô hình LSTM và GRU được huấn luyện trên tập dữ liệu kết hợp với cùng cấu hình như trong chiến lược 1.
    \item Với hai loại dữ liệu đầu vào (chỉ lưu lượng và lưu lượng + áp suất), tổng cộng sẽ có 4 mô hình được huấn luyện (1 tập dữ liệu kết hợp × 2 kiến trúc × 2 loại dữ liệu đầu vào).
\end{itemize}

Chiến lược kép này không chỉ cho phép đánh giá hiệu suất của các kiến trúc mô hình khác nhau mà còn cung cấp cái nhìn sâu sắc về tính hiệu quả của việc huấn luyện mô hình chuyên biệt so với mô hình tổng quát trong bối cảnh dự báo lưu lượng nước và phát hiện bất thường. Kết quả của quá trình huấn luyện và đánh giá hiệu suất sẽ được trình bày chi tiết trong phần tiếp theo.

\subsection{Đánh giá mô hình}

Dựa trên kết quả huấn luyện từ hai chiến lược khác nhau, chúng tôi tiến hành phân tích chi tiết hiệu suất của các mô hình LSTM và GRU trong việc dự báo lưu lượng nước. Kết quả được đánh giá thông qua hai chỉ số chính: RMSE (Root Mean Square Error) và MAE (Mean Absolute Error), phản ánh độ chính xác của các dự báo.

\subsubsection{Đánh giá theo từng điểm đo}

Bảng \ref{tab:model_performance_by_location} trình bày kết quả đánh giá hiệu suất của các mô hình tại từng điểm đo riêng biệt.

\begin{table}[htbp]
    \centering
    \begin{tabular}{|l|l|c|c|c|c|}
        \hline
        \textbf{Điểm đo} & \textbf{Dữ liệu đầu vào} & \multicolumn{2}{c|}{\textbf{LSTM}} & \multicolumn{2}{c|}{\textbf{GRU}} \\
        \cline{3-6}
        & & \textbf{RMSE} & \textbf{MAE} & \textbf{RMSE} & \textbf{MAE} \\
        \hline
        \multirow{2}{*}{841211914190} & Chỉ lưu lượng & 31.0548 & 17.6299 & 31.4756 & 17.7724 \\
        \cline{2-6}
        & Lưu lượng + áp suất & 30.0085 & 18.0322 & 30.0749 & 17.4541 \\
        \hline
        \multirow{2}{*}{8401210607558} & Chỉ lưu lượng & 45.5464 & 31.3453 & 45.4384 & 31.3374 \\
        \cline{2-6}
        & Lưu lượng + áp suất & 45.7258 & 31.7983 & 48.1915 & 33.1949 \\
        \hline
    \end{tabular}
    \caption{Hiệu suất của các mô hình theo từng điểm đo}
    \label{tab:model_performance_by_location}
\end{table}

Từ bảng kết quả, có thể thấy tại điểm đo 841211914190, việc bổ sung thêm dữ liệu áp suất đã cải thiện nhẹ hiệu suất của cả hai mô hình (giảm RMSE từ 31.0548 xuống 30.0085 đối với LSTM). Điều này phù hợp với lý thuyết về mối quan hệ thủy lực giữa áp suất và lưu lượng trong hệ thống cấp nước.

Tuy nhiên, tại điểm đo 8401210607558, việc bổ sung thêm dữ liệu áp suất không mang lại cải thiện đáng kể cho hiệu suất mô hình. Thực tế, mô hình GRU còn cho thấy sự suy giảm hiệu suất khi sử dụng thêm dữ liệu áp suất (RMSE tăng từ 45.4384 lên 48.1915).

\subsubsection{Đánh giá mô hình tổng hợp}

Bảng \ref{tab:combined_model_performance} trình bày kết quả đánh giá hiệu suất của các mô hình khi được huấn luyện trên tập dữ liệu kết hợp từ cả hai điểm đo.

\begin{table}[htbp]
    \centering
    \begin{tabular}{|l|c|c|c|c|}
        \hline
        \textbf{Dữ liệu đầu vào} & \multicolumn{2}{c|}{\textbf{LSTM}} & \multicolumn{2}{c|}{\textbf{GRU}} \\
        \cline{2-5}
        & \textbf{RMSE} & \textbf{MAE} & \textbf{RMSE} & \textbf{MAE} \\
        \hline
        Chỉ lưu lượng & 42.7537 & 28.5852 & 42.7337 & 28.5339 \\
        \hline
        Lưu lượng + áp suất & 43.5561 & 28.7302 & 45.0266 & 29.7936 \\
        \hline
    \end{tabular}
    \caption{Hiệu suất của các mô hình tổng hợp}
    \label{tab:combined_model_performance}
\end{table}

\subsubsection{Phân tích và thảo luận}

Từ kết quả trên, có thể rút ra một số nhận xét quan trọng:

\begin{enumerate}
    \item \textbf{Hiệu suất tương đối giữa LSTM và GRU:} Cả hai kiến trúc mạng đều thể hiện hiệu suất tương đương trong hầu hết các trường hợp, với sự khác biệt không đáng kể về RMSE và MAE. Điều này phù hợp với các nghiên cứu trước đây về khả năng xử lý chuỗi thời gian của hai kiến trúc này.
    
    \item \textbf{Tác động của việc bổ sung dữ liệu áp suất:} Việc bổ sung thêm dữ liệu áp suất không phải lúc nào cũng mang lại cải thiện hiệu suất. Tại điểm đo 841211914190, việc bổ sung dữ liệu áp suất có tác động tích cực, trong khi tại điểm đo 8401210607558, tác động này không rõ ràng và thậm chí có thể gây suy giảm hiệu suất.
    
    \item \textbf{Chiến lược huấn luyện:} Mô hình được huấn luyện riêng cho từng điểm đo thường cho hiệu suất tốt hơn so với mô hình tổng hợp, đặc biệt là tại điểm đo 841211914190. Điều này cho thấy việc nắm bắt đặc thù riêng của từng điểm đo có thể quan trọng hơn việc học các mẫu chung từ toàn bộ hệ thống.
\end{enumerate}

Bảng \ref{tab:model_performance_by_location} và \ref{tab:combined_model_performance} cung cấp cái nhìn tổng quan rõ ràng về hiệu suất của các mô hình, giúp dễ dàng so sánh và đánh giá các chiến lược huấn luyện khác nhau. Kết quả này cung cấp cơ sở cho việc lựa chọn chiến lược huấn luyện và kiến trúc mô hình phù hợp cho bài toán dự báo lưu lượng nước trong hệ thống cấp nước. Việc cân nhắc giữa độ phức tạp của mô hình và hiệu suất dự báo là yếu tố quan trọng trong quá trình triển khai thực tế.


\section{Phát hiện bất thường}
\subsection{Tổng quan}
% TODO: tạo lược đồ tại đây

\subsection{Phát hiện bằng sai số}
Trong phần này, chúng tôi trình bày phương pháp phát hiện bất thường dựa trên sai số dự đoán của các mô hình đã được huấn luyện. Phương pháp này dựa trên nguyên tắc cơ bản về thống kê mà theo đó, trong điều kiện bình thường, sai số dự đoán của mô hình thường tuân theo phân phối xác suất nhất định, và các giá trị nằm ngoài khoảng tin cậy có thể được xem là bất thường.

\subsubsection{Cơ sở lý thuyết}
Như đã trình bày trong Phần 3.2, dữ liệu bất thường (anomalies hoặc outliers) là các quan sát khác biệt đáng kể so với mẫu hành vi thông thường của chuỗi thời gian. Một giá trị \( y_t \) được coi là bất thường nếu:

\begin{equation}
    |y_t - \hat{y}_t| > \delta,
\end{equation}

trong đó \(\hat{y}_t\) là giá trị dự đoán từ mô hình, và \(\delta\) là ngưỡng xác định dựa trên phân phối của dữ liệu.

Trong nghiên cứu này, chúng tôi chọn ngưỡng \(\delta\) bằng ba lần độ lệch chuẩn của sai số dự đoán:

\begin{equation}
\delta = 3 \times \sigma_e
\end{equation}

trong đó \(\sigma_e\) là độ lệch chuẩn của sai số dự đoán trong tập huấn luyện, được tính bằng:

\begin{equation}
\sigma_e = \sqrt{\frac{1}{N} \sum_{i=1}^{N} (y_i - \hat{y}_i)^2}
\end{equation}

Việc sử dụng ngưỡng ba lần độ lệch chuẩn dựa trên quy tắc 3-sigma, một nguyên lý thống kê được áp dụng rộng rãi trong phát hiện bất thường \cite{malhotra2016lstm, hundman2018detecting}. Theo quy tắc này, đối với dữ liệu tuân theo phân phối chuẩn, khoảng 99.7\% các quan sát sẽ nằm trong phạm vi ±3 độ lệch chuẩn từ giá trị trung bình. Do đó, bất kỳ quan sát nào vượt ra ngoài khoảng này có xác suất rất thấp (khoảng 0.3\%) để xảy ra trong điều kiện bình thường, và có thể được xem là bất thường.

\subsubsection{Quy trình thực hiện}
Quy trình phát hiện bất thường bằng phương pháp sai số được thực hiện theo các bước sau:

\begin{enumerate}
    \item Sử dụng các mô hình đã huấn luyện (LSTM và GRU với các chiến lược khác nhau) để dự đoán lưu lượng nước trên tập dữ liệu kiểm tra.
    \item Tính toán sai số giữa giá trị dự đoán và giá trị thực tế: \(e_t = y_t - \hat{y}_t\).
    \item Tính độ lệch chuẩn của sai số trên tập huấn luyện: \(\sigma_e\).
    \item Xác định ngưỡng phát hiện bất thường: \(\delta = 3 \times \sigma_e\).
    \item Đánh dấu các điểm dữ liệu là bất thường nếu \(|e_t| > \delta\).
\end{enumerate}

Quy trình này được áp dụng cho tất cả mười hai mô hình đã được huấn luyện trong phần trước: bao gồm các mô hình LSTM và GRU cho từng điểm đo riêng biệt và mô hình tổng hợp, với hai loại dữ liệu đầu vào khác nhau (chỉ lưu lượng và kết hợp lưu lượng với áp suất). Việc sử dụng nhiều mô hình khác nhau giúp đánh giá độ tin cậy và tính nhất quán của phương pháp này trong việc phát hiện bất thường.

\subsubsection{Kết quả phát hiện bất thường}

\begin{figure}[htbp]
    \centering
    \includegraphics[width=\textwidth]{image/section6_3/anomaly_detection_841211914190_lstm_flow.png}
    \caption{Kết quả phát hiện bất thường bằng mô hình LSTM với dữ liệu lưu lượng tại điểm đo 841211914190. Đường màu xanh thể hiện giá trị thực tế, đường màu cam thể hiện giá trị dự đoán, và các điểm màu đỏ biểu thị các điểm dữ liệu được xác định là bất thường.}
    \label{fig:anomaly_lstm_841211914190_flow}
\end{figure}

\begin{figure}[htbp]
    \centering
    \includegraphics[width=\textwidth]{image/section6_3/anomaly_detection_841211914190_lstm_allmetric.png}
    \caption{Kết quả phát hiện bất thường bằng mô hình LSTM với dữ liệu kết hợp lưu lượng và áp suất tại điểm đo 841211914190.}
    \label{fig:anomaly_lstm_841211914190_all}
\end{figure}

\begin{figure}[htbp]
    \centering
    \includegraphics[width=\textwidth]{image/section6_3/anomaly_detection_841211914190_gru_flow.png}
    \caption{Kết quả phát hiện bất thường bằng mô hình GRU với dữ liệu lưu lượng tại điểm đo 841211914190.}
    \label{fig:anomaly_gru_841211914190_flow}
\end{figure}

\begin{figure}[htbp]
    \centering
    \includegraphics[width=\textwidth]{image/section6_3/anomaly_detection_841211914190_gru_allmetric.png}
    \caption{Kết quả phát hiện bất thường bằng mô hình GRU với dữ liệu kết hợp lưu lượng và áp suất tại điểm đo 841211914190.}
    \label{fig:anomaly_gru_841211914190_all}
\end{figure}

\begin{figure}[htbp]
    \centering
    \includegraphics[width=\textwidth]{image/section6_3/anomaly_detection_8401210607558_lstm_flow.png}
    \caption{Kết quả phát hiện bất thường bằng mô hình LSTM với dữ liệu lưu lượng tại điểm đo 8401210607558.}
    \label{fig:anomaly_lstm_8401210607558_flow}
\end{figure}

\begin{figure}[htbp]
    \centering
    \includegraphics[width=\textwidth]{image/section6_3/anomaly_detection_8401210607558_lstm_all_metric.png}
    \caption{Kết quả phát hiện bất thường bằng mô hình LSTM với dữ liệu kết hợp lưu lượng và áp suất tại điểm đo 8401210607558.}
    \label{fig:anomaly_lstm_8401210607558_all}
\end{figure}

\begin{figure}[htbp]
    \centering
    \includegraphics[width=\textwidth]{image/section6_3/anomaly_detection_8401210607558_gru_flow.png}
    \caption{Kết quả phát hiện bất thường bằng mô hình GRU với dữ liệu lưu lượng tại điểm đo 8401210607558.}
    \label{fig:anomaly_gru_8401210607558_flow}
\end{figure}

\begin{figure}[htbp]
    \centering
    \includegraphics[width=\textwidth]{image/section6_3/anomaly_detection_8401210607558_gru_allmetric.png}
    \caption{Kết quả phát hiện bất thường bằng mô hình GRU với dữ liệu kết hợp lưu lượng và áp suất tại điểm đo 8401210607558.}
    \label{fig:anomaly_gru_8401210607558_all}
\end{figure}

\begin{figure}[htbp]
    \centering
    \includegraphics[width=\textwidth]{image/section6_3/anomaly_detection_combined_lstm_flow.png}
    \caption{Kết quả phát hiện bất thường bằng mô hình LSTM tổng hợp với dữ liệu lưu lượng.}
    \label{fig:anomaly_combined_lstm_flow}
\end{figure}

\begin{figure}[htbp]
    \centering
    \includegraphics[width=\textwidth]{image/section6_3/anomaly_detection_combined_lstm_allmetric.png}
    \caption{Kết quả phát hiện bất thường bằng mô hình LSTM tổng hợp với dữ liệu kết hợp lưu lượng và áp suất.}
    \label{fig:anomaly_combined_lstm_all}
\end{figure}

\begin{figure}[htbp]
    \centering
    \includegraphics[width=\textwidth]{image/section6_3/anomaly_detection_combined_gru_flow.png}
    \caption{Kết quả phát hiện bất thường bằng mô hình GRU tổng hợp với dữ liệu lưu lượng.}
    \label{fig:anomaly_combined_gru_flow}
\end{figure}

\begin{figure}[htbp]
    \centering
    \includegraphics[width=\textwidth]{image/section6_3/anomaly_detection_combined_gru_allmetric.png}
    \caption{Kết quả phát hiện bất thường bằng mô hình GRU tổng hợp với dữ liệu kết hợp lưu lượng và áp suất.}
    \label{fig:anomaly_combined_gru_all}
\end{figure}

\subsubsection{Phân tích kết quả phát hiện bất thường}

Từ kết quả phát hiện bất thường trên các hình từ \ref{fig:anomaly_lstm_841211914190_flow} đến \ref{fig:anomaly_combined_gru_all}, chúng tôi rút ra một số nhận xét quan trọng:

\begin{enumerate}
    \item \textbf{Đặc điểm của các điểm bất thường:} Các điểm bất thường thường xuất hiện trong các khoảng thời gian có biến động lớn về lưu lượng, đặc biệt là các đỉnh tiêu thụ cao hoặc các thời điểm chuyển tiếp giữa tiêu thụ thấp và cao. Điều này phù hợp với đặc điểm dữ liệu chuỗi thời gian đã được mô tả trong Phần 3.2, trong đó các bất thường thường nổi bật hơn khi so sánh với mẫu hành vi điển hình.
    
    \item \textbf{Phân bố thời gian của các bất thường:} Các điểm bất thường không phân bố đều trong thời gian mà thường xuất hiện theo cụm, đặc biệt vào các thời điểm có sự thay đổi đột ngột trong lưu lượng nước. Điều này phản ánh khả năng có các sự kiện hoặc hiện tượng đặc biệt xảy ra trong những khoảng thời gian đó, có thể liên quan đến rò rỉ, sự cố van, hoặc tiêu thụ bất thường.
    
    \item \textbf{So sánh giữa các mô hình:} Mô hình LSTM và GRU cho kết quả phát hiện bất thường tương đối giống nhau, nhưng có một số điểm khác biệt, đặc biệt trong các khoảng thời gian có biến động phức tạp. Điều này cho thấy mỗi kiến trúc mạng có những điểm mạnh riêng trong việc nắm bắt các mẫu hành vi khác nhau trong dữ liệu.
    
    \item \textbf{Độ nhạy của mô hình:} Mô hình được huấn luyện riêng cho từng điểm đo thường nhạy hơn trong việc phát hiện các bất thường cục bộ, trong khi mô hình tổng hợp có xu hướng phát hiện các bất thường chung hơn. Điều này phù hợp với kết quả đánh giá hiệu suất mô hình đã được trình bày trong phần trước.
    
    \item \textbf{Tương quan với các hiện tượng ban đêm:} Đáng chú ý, một số điểm bất thường được phát hiện trong khoảng thời gian ban đêm (từ 00:00 đến 05:00), khi lưu lượng nước thường ổn định và thấp hơn. Điều này phù hợp với nhận định trong Phần 3.2 về việc biến động dữ liệu vào ban đêm và khả năng phát hiện bất thường rõ ràng hơn trong khoảng thời gian này. Các bất thường này có thể là dấu hiệu của rò rỉ, vì rò rỉ thường nổi bật hơn khi tiêu thụ bình thường thấp.
\end{enumerate}

\begin{table}[htbp]
    \centering
    \begin{tabular}{|l|l|c|c|c|}
        \hline
        \textbf{Mô hình} & \textbf{Dữ liệu đầu vào} & \textbf{Số điểm bất thường} & \textbf{Tổng số điểm} & \textbf{Tỷ lệ (\%)} \\
        \hline
        \multirow{2}{*}{LSTM - 841211914190} & Chỉ lưu lượng & 128 & 4.752 & 2,69 \\
        \cline{2-5}
        & Lưu lượng + áp suất & 135 & 4.752 & 2,84 \\
        \hline
        \multirow{2}{*}{GRU - 841211914190} & Chỉ lưu lượng & 132 & 4.752 & 2,78 \\
        \cline{2-5}
        & Lưu lượng + áp suất & 130 & 4.752 & 2,74 \\
        \hline
        \multirow{2}{*}{LSTM - 8401210607558} & Chỉ lưu lượng & 153 & 6.816 & 2,24 \\
        \cline{2-5}
        & Lưu lượng + áp suất & 161 & 6.816 & 2,36 \\
        \hline
        \multirow{2}{*}{GRU - 8401210607558} & Chỉ lưu lượng & 166 & 6.816 & 2,44 \\
        \cline{2-5}
        & Lưu lượng + áp suất & 174 & 6.816 & 2,55 \\
        \hline
        \multirow{2}{*}{LSTM - Tổng hợp} & Chỉ lưu lượng & 218 & 11.568 & 1,88 \\
        \cline{2-5}
        & Lưu lượng + áp suất & 225 & 11.568 & 1,95 \\
        \hline
        \multirow{2}{*}{GRU - Tổng hợp} & Chỉ lưu lượng & 231 & 11.568 & 2,00 \\
        \cline{2-5}
        & Lưu lượng + áp suất & 240 & 11.568 & 2,07 \\
        \hline
    \end{tabular}
    \caption{Tổng hợp số lượng điểm bất thường được phát hiện bởi các mô hình}
    \label{tab:anomaly_detection_summary}
\end{table}

Tỷ lệ điểm bất thường được phát hiện nằm trong khoảng từ 1,88\% đến 2,84\%, cao hơn một chút so với kỳ vọng lý thuyết 0,3\% dựa trên quy tắc 3-sigma. Điều này có thể được giải thích bởi một số yếu tố:

\begin{itemize}
    \item Phân phối sai số dự đoán có thể không hoàn toàn tuân theo phân phối chuẩn, mà có thể có đuôi dày hơn (heavy-tailed)
    \item Sự hiện diện của các hiện tượng bất thường thực sự trong dữ liệu thực tế, bao gồm rò rỉ, sự cố thiết bị, hoặc các thay đổi đột ngột trong mẫu tiêu thụ
    \item Khả năng của mô hình trong việc nắm bắt các mẫu hành vi phức tạp và phi tuyến trong dữ liệu thủy lực
\end{itemize}

Phân tích chi tiết về phân bố thời gian của các điểm bất thường cho thấy chúng thường xuất hiện trong các khoảng thời gian cụ thể, điều này gợi ý rằng chúng không phải là các phát hiện ngẫu nhiên mà có thể liên quan đến các sự kiện hoặc hiện tượng thực tế.

\subsubsection{Kết luận về phương pháp phát hiện bất thường dựa trên sai số}

Phương pháp phát hiện bất thường dựa trên ngưỡng 3 lần độ lệch chuẩn đã chứng minh được hiệu quả trong việc xác định các điểm dữ liệu có hành vi khác thường trong chuỗi thời gian lưu lượng nước. Phương pháp này có một số ưu điểm đáng chú ý:

\begin{itemize}
    \item \textbf{Tính đơn giản và hiệu quả:} Phương pháp này dễ dàng triển khai và không đòi hỏi quá trình huấn luyện bổ sung sau khi đã có mô hình dự báo cơ sở.
    
    \item \textbf{Cơ sở thống kê vững chắc:} Dựa trên nguyên lý thống kê được chấp nhận rộng rãi (quy tắc 3-sigma), phương pháp này cung cấp một cách tiếp cận khách quan để xác định các điểm dữ liệu bất thường.
    
    \item \textbf{Kết quả trực quan:} Các điểm bất thường được xác định rõ ràng trên biểu đồ chuỗi thời gian, giúp dễ dàng theo dõi và phân tích.
    
    \item \textbf{Khả năng tùy chỉnh:} Ngưỡng phát hiện có thể được điều chỉnh (ví dụ: 2σ, 2,5σ, hoặc 3,5σ) tùy theo mức độ nhạy cảm mong muốn và đặc điểm cụ thể của dữ liệu.
\end{itemize}

Tuy nhiên, phương pháp này cũng có một số hạn chế cần lưu ý:

\begin{itemize}
    \item \textbf{Giả định về phân phối:} Phương pháp này đạt hiệu quả tối ưu khi sai số dự đoán tuân theo phân phối chuẩn, điều này không phải lúc nào cũng đúng trong thực tế.
    
    \item \textbf{Nhạy cảm với chất lượng mô hình:} Hiệu quả phát hiện bất thường phụ thuộc vào chất lượng của mô hình dự báo cơ sở. Một mô hình dự báo kém có thể tạo ra nhiều sai số lớn, dẫn đến tỷ lệ báo động giả cao.
    
    \item \textbf{Khó khăn trong việc phát hiện bất thường dần dần:} Các bất thường phát triển chậm (như rò rỉ nhỏ phát triển dần) có thể không tạo ra sai số đủ lớn để vượt qua ngưỡng 3σ trong giai đoạn đầu.
\end{itemize}

Trong các phần tiếp theo, chúng tôi sẽ khám phá các phương pháp học không giám sát khác để phát hiện bất thường, và so sánh hiệu quả của chúng với phương pháp dựa trên sai số đã trình bày.
\subsection{Phát hiện bằng các phương pháp học không giám sát}
\subsubsection{Isolation Forest}
\subsubsection{LOF}
\subsubsection{One-class SVM}
\subsection{Phân tích kết quả}

\subsection{Phát hiện bằng các phương pháp học không giám sát}

Ngoài phương pháp dựa trên sai số dự đoán từ mô hình học sâu, chúng tôi tiếp tục khám phá khả năng phát hiện bất thường thông qua các phương pháp học không giám sát. Ưu điểm chính của các phương pháp này là khả năng hoạt động mà không cần dữ liệu được gán nhãn trước, cho phép phát hiện các dạng bất thường chưa biết trước. Các thuật toán không giám sát này có thể trực tiếp học từ cấu trúc dữ liệu để xác định các mẫu khác biệt so với phần lớn dữ liệu.

\subsubsection{Isolation Forest}

Isolation Forest (IF) là thuật toán phát hiện bất thường dựa trên nguyên lý rằng các điểm dữ liệu bất thường thường dễ dàng bị "cô lập" hơn so với các điểm dữ liệu bình thường. Thuật toán xây dựng các cây quyết định ngẫu nhiên và đánh giá độ bất thường dựa trên số lượng phân chia cần thiết để cô lập một điểm dữ liệu.

\paragraph{Nguyên lý hoạt động}
Isolation Forest hoạt động thông qua các bước sau:
\begin{enumerate}
    \item Xây dựng một tập hợp các cây quyết định (Isolation Trees).
    \item Tại mỗi nút của cây, thuật toán chọn ngẫu nhiên một thuộc tính và một ngưỡng phân chia.
    \item Quá trình phân chia tiếp tục cho đến khi mỗi điểm dữ liệu được cô lập hoàn toàn.
    \item Điểm bất thường được xác định dựa trên độ sâu trung bình cần thiết để cô lập điểm đó - các điểm cần ít sự phân chia hơn (độ sâu cây thấp hơn) được coi là bất thường.
\end{enumerate}

Đối với bộ dữ liệu chuỗi thời gian của chúng tôi, Isolation Forest được áp dụng trên ma trận đặc trưng bao gồm ba thông số: lưu lượng, áp suất đầu vào và áp suất đầu ra. Mỗi điểm dữ liệu trong không gian ba chiều này được đánh giá mức độ bất thường dựa trên khả năng bị cô lập.

\paragraph{Triển khai và kết quả}
Chúng tôi triển khai Isolation Forest với các tham số sau:
\begin{itemize}
    \item Số lượng cây (n\_estimators): 100
    \item Tỷ lệ mẫu con (contamination): được ước tính là 0.05 dựa trên phân tích sơ bộ
    \item Số lượng mẫu tối đa cho mỗi cây (max\_samples): "auto" (tối đa 256 mẫu)
\end{itemize}

Sau khi áp dụng Isolation Forest trên toàn bộ tập dữ liệu, thuật toán đã phát hiện được 1.682 điểm bất thường, chiếm 4,94\% tổng số điểm dữ liệu. Hình \ref{fig:isolation_forest_results} minh họa kết quả phát hiện bất thường của Isolation Forest trên chuỗi thời gian.

% \begin{figure}[htbp]
%     \centering
%     \includegraphics[width=\textwidth]{image/section6_3/isolation_forest_results.png}
%     \caption{Kết quả phát hiện bất thường bằng Isolation Forest. Các điểm màu đỏ biểu thị các điểm được xác định là bất thường.}
%     \label{fig:isolation_forest_results}
% \end{figure}

Như có thể thấy từ hình trên, Isolation Forest đặc biệt nhạy với các điểm dữ liệu có sự biến động mạnh, đặc biệt là các đỉnh và đáy cực trị trong chuỗi lưu lượng. Đây là đặc tính phổ biến của thuật toán này, phù hợp với việc phát hiện các bất thường dạng điểm (point anomalies) và các biến động đột ngột, điều này đặc biệt hữu ích trong việc phát hiện sự cố rò rỉ đột ngột hoặc sự thay đổi lớn trong mẫu tiêu thụ nước.

\subsubsection{Local Outlier Factor (LOF)}

Phương pháp Local Outlier Factor (LOF) dựa trên khái niệm mật độ cục bộ để phát hiện bất thường. Thay vì xem xét toàn bộ không gian dữ liệu, LOF tập trung vào mật độ tương đối của một điểm dữ liệu so với các điểm lân cận của nó.

\paragraph{Nguyên lý hoạt động}
LOF hoạt động thông qua các bước chính sau:
\begin{enumerate}
    \item Xác định k điểm lân cận gần nhất cho mỗi điểm dữ liệu.
    \item Tính toán mật độ cục bộ cho mỗi điểm dựa trên khoảng cách đến các điểm lân cận.
    \item Tính toán hệ số LOF bằng cách so sánh mật độ cục bộ của điểm đó với mật độ cục bộ của các điểm lân cận.
    \item Điểm có hệ số LOF cao (mật độ thấp hơn đáng kể so với các lân cận) được coi là bất thường.
\end{enumerate}

Ưu điểm của LOF là khả năng phát hiện bất thường cục bộ - các điểm dữ liệu có thể không phải là bất thường trong toàn bộ tập dữ liệu nhưng bất thường so với vùng lân cận của chúng. Đặc điểm này đặc biệt hữu ích cho dữ liệu có cấu trúc phức tạp hoặc nhiều nhóm với mật độ khác nhau.

\paragraph{Triển khai và kết quả}
Chúng tôi áp dụng LOF với các tham số:
\begin{itemize}
    \item Số lượng láng giềng (n\_neighbors): 20
    \item Thuật toán tìm kiếm láng giềng: "auto" (tự động chọn thuật toán tối ưu)
    \item Tỷ lệ mẫu bất thường (contamination): 0.05
\end{itemize}

Sau khi áp dụng LOF trên toàn bộ tập dữ liệu, thuật toán đã phát hiện 273 điểm bất thường, chiếm 0,80\% tổng số điểm dữ liệu. Con số này thấp hơn đáng kể so với Isolation Forest, phản ánh tính chọn lọc cao hơn của LOF trong việc xác định các bất thường thực sự dựa trên cấu trúc mật độ cục bộ.

% \begin{figure}[htbp]
%     \centering
%     \includegraphics[width=\textwidth]{image/section6_3/lof_results.png}
%     \caption{Kết quả phát hiện bất thường bằng Local Outlier Factor. Các điểm màu đỏ biểu thị các điểm được xác định là bất thường.}
%     \label{fig:lof_results}
% \end{figure}

Như minh họa trong Hình \ref{fig:lof_results}, LOF phát hiện các điểm bất thường tập trung chủ yếu vào các vị trí có sự thay đổi đột ngột về mật độ dữ liệu, đặc biệt là các điểm có giá trị cao bất thường so với các điểm xung quanh. Điều đáng chú ý là LOF phát hiện ít bất thường hơn nhiều so với Isolation Forest, phản ánh sự khác biệt cơ bản trong định nghĩa "bất thường" giữa hai thuật toán. LOF tập trung vào các điểm thực sự khác biệt so với các điểm lân cận, trong khi Isolation Forest có xu hướng xác định nhiều điểm hơn do cách tiếp cận phân vùng ngẫu nhiên.

\subsubsection{One-class SVM}

One-class Support Vector Machine (OCSVM) là một biến thể của thuật toán SVM truyền thống, được thiết kế đặc biệt cho bài toán phát hiện bất thường. Thay vì phân loại dữ liệu thành các lớp khác nhau, OCSVM tìm cách xác định ranh giới bao quanh phần lớn dữ liệu bình thường và xác định các điểm nằm ngoài ranh giới này là bất thường.

\paragraph{Nguyên lý hoạt động}
One-class SVM hoạt động dựa trên các nguyên tắc sau:
\begin{enumerate}
    \item Ánh xạ dữ liệu vào không gian đặc trưng có chiều cao hơn thông qua hàm kernel.
    \item Tìm một siêu phẳng tối ưu trong không gian mới này sao cho nó tách biệt phần lớn dữ liệu với gốc tọa độ và tối đa hóa khoảng cách từ siêu phẳng đến gốc tọa độ.
    \item Sử dụng một tham số $\nu$ để kiểm soát tỷ lệ điểm dữ liệu có thể nằm ngoài siêu phẳng (tức là tỷ lệ bất thường).
    \item Các điểm nằm ngoài ranh giới được xác định bởi siêu phẳng được coi là bất thường.
\end{enumerate}

One-class SVM đặc biệt hiệu quả khi dữ liệu bình thường có thể được mô tả bằng một ranh giới đơn lẻ trong không gian đặc trưng, và có khả năng phát hiện các bất thường phức tạp nhờ khả năng ánh xạ phi tuyến tính của các hàm kernel.

\paragraph{Triển khai và kết quả}
Chúng tôi triển khai One-class SVM với các tham số:
\begin{itemize}
    \item Kernel: RBF (Radial Basis Function)
    \item Nu ($\nu$): 0.05
    \item Gamma: "scale" (1 / (n\_features * X.var()))
\end{itemize}

Sau khi áp dụng One-class SVM, thuật toán đã phát hiện 1.624 điểm bất thường, chiếm 4,77\% tổng số điểm dữ liệu. Kết quả này khá tương đồng với Isolation Forest về số lượng bất thường được phát hiện, mặc dù có sự khác biệt về vị trí cụ thể của các điểm bất thường.

% \begin{figure}[htbp]
%     \centering
%     \includegraphics[width=\textwidth]{image/section6_3/ocsvm_results.png}
%     \caption{Kết quả phát hiện bất thường bằng One-class SVM. Các điểm màu đỏ biểu thị các điểm được xác định là bất thường.}
%     \label{fig:ocsvm_results}
% \end{figure}

Như thể hiện trong Hình \ref{fig:ocsvm_results}, One-class SVM phát hiện các bất thường tập trung chủ yếu vào các vùng có giá trị cực trị và các điểm có sự biến động lớn so với xu hướng chung của dữ liệu. Đặc biệt, thuật toán này nhạy với các điểm có giá trị cao bất thường trong chuỗi lưu lượng, điều này gợi ý khả năng phát hiện các sự cố tiêu thụ đột biến hoặc rò rỉ lớn.

\subsection{Phân tích kết quả}

Sau khi áp dụng bốn phương pháp phát hiện bất thường khác nhau (dựa trên sai số dự đoán, Isolation Forest, LOF và One-class SVM), chúng tôi tiến hành phân tích so sánh kết quả để đánh giá hiệu quả tương đối của từng phương pháp và tìm hiểu về đặc điểm của các bất thường được phát hiện.

\begin{table}[htbp]
    \centering
    \begin{tabular}{|l|c|c|c|}
        \hline
        \textbf{Phương pháp} & \textbf{Số lượng bất thường} & \textbf{Tổng số điểm} & \textbf{Tỷ lệ (\%)} \\
        \hline
        Phương pháp dựa trên sai số (LSTM) & 225 & 11.568 & 1,95 \\
        \hline
        Isolation Forest & 1.682 & 34.080 & 4,94 \\
        \hline
        Local Outlier Factor & 273 & 34.080 & 0,80 \\
        \hline
        One-class SVM & 1.624 & 34.080 & 4,77 \\
        \hline
    \end{tabular}
    \caption{So sánh số lượng bất thường được phát hiện bởi các phương pháp khác nhau}
    \label{tab:anomaly_methods_comparison}
\end{table}

Từ Bảng \ref{tab:anomaly_methods_comparison}, có thể thấy sự khác biệt đáng kể về số lượng bất thường được phát hiện bởi các phương pháp khác nhau. Isolation Forest và One-class SVM phát hiện tỷ lệ bất thường cao nhất (lần lượt là 4,94\% và 4,77\%), trong khi LOF chỉ phát hiện 0,80\% số điểm là bất thường. Phương pháp dựa trên sai số dự đoán từ mô hình LSTM nằm ở mức trung bình với 1,95\%.

% \begin{figure}[htbp]
%     \centering
%     \includegraphics[width=\textwidth]{image/section6_3/combined_anomaly_detection.png}
%     \caption{So sánh kết quả phát hiện bất thường từ bốn phương pháp. Mỗi hàng biểu thị một phương pháp khác nhau, với các điểm màu đỏ thể hiện bất thường được phát hiện.}
%     \label{fig:combined_anomaly_detection}
% \end{figure}

Sự khác biệt này phản ánh đặc điểm cơ bản của từng thuật toán:

\begin{itemize}
    \item \textbf{Isolation Forest và One-class SVM} có xu hướng phát hiện nhiều bất thường hơn do cách tiếp cận tổng thể của chúng. Isolation Forest dễ dàng cô lập các điểm có giá trị cực trị, trong khi One-class SVM nhạy với các điểm nằm ngoài ranh giới tổng quát của dữ liệu bình thường.
    
    \item \textbf{Local Outlier Factor} thể hiện tính chọn lọc cao nhất, chỉ phát hiện các điểm thực sự khác biệt về mật độ so với vùng lân cận của chúng. Điều này phù hợp với định nghĩa bất thường cục bộ, nơi một điểm được coi là bất thường chỉ khi nó khác biệt đáng kể so với các điểm xung quanh, không phải so với toàn bộ tập dữ liệu.
    
    \item \textbf{Phương pháp dựa trên sai số} từ mô hình LSTM có độ nhạy trung bình, tập trung vào các điểm mà mô hình không thể dự đoán chính xác. Phương pháp này kết hợp cả hiểu biết về mẫu thời gian và các biến động bất thường.
\end{itemize}

Ngoài ra, chúng tôi cũng phân tích mức độ chồng lấp giữa các bất thường được phát hiện bởi các phương pháp khác nhau.

\begin{table}[htbp]
    \centering
    \begin{tabular}{|l|c|c|c|c|}
        \hline
        \textbf{Phương pháp} & \textbf{Dựa trên sai số} & \textbf{Isolation Forest} & \textbf{LOF} & \textbf{One-class SVM} \\
        \hline
        Dựa trên sai số & 225 & 153 & 32 & 149 \\
        \hline
        Isolation Forest & 153 & 1.682 & 204 & 1.146 \\
        \hline
        LOF & 32 & 204 & 273 & 198 \\
        \hline
        One-class SVM & 149 & 1.146 & 198 & 1.624 \\
        \hline
    \end{tabular}
    \caption{Ma trận chồng lấp giữa các bất thường được phát hiện bởi các phương pháp khác nhau}
    \label{tab:anomaly_overlap_matrix}
\end{table}

Từ Bảng \ref{tab:anomaly_overlap_matrix}, có thể thấy mức độ chồng lấp cao giữa Isolation Forest và One-class SVM (1.146 điểm chung), phản ánh sự tương đồng trong cách định nghĩa bất thường của hai phương pháp này. Ngược lại, LOF có mức độ chồng lấp thấp nhất với phương pháp dựa trên sai số (chỉ 32 điểm chung), điều này cho thấy sự khác biệt đáng kể trong định nghĩa bất thường giữa hai phương pháp này.

Phân tích đặc điểm thời gian của các bất thường cũng mang lại những hiểu biết quan trọng:

\begin{itemize}
    \item \textbf{Bất thường ban đêm}: Cả bốn phương pháp đều phát hiện một số bất thường trong khoảng thời gian ban đêm (00:00-05:00), mặc dù tỷ lệ khác nhau. Isolation Forest và One-class SVM phát hiện nhiều bất thường ban đêm hơn, trong khi LOF tập trung vào các bất thường cục bộ mà không phụ thuộc nhiều vào thời gian trong ngày.
    
    \item \textbf{Bất thường theo mùa}: Phân tích theo tháng cho thấy tỷ lệ bất thường cao hơn trong các tháng mùa hè và mùa đông cực đoan, gợi ý mối liên hệ với mẫu tiêu thụ nước theo mùa và khả năng ảnh hưởng của nhiệt độ đến hành vi tiêu thụ và hiệu suất hệ thống.
    
    \item \textbf{Bất thường dạng cụm}: Nhiều bất thường xuất hiện theo cụm liên tiếp, đặc biệt trong kết quả từ Isolation Forest và One-class SVM. Điều này gợi ý sự hiện diện của các sự kiện bất thường kéo dài như rò rỉ, thay vì chỉ là các điểm dữ liệu nhiễu ngẫu nhiên.
\end{itemize}

Điểm quan trọng cần lưu ý là mỗi phương pháp phát hiện các dạng bất thường khác nhau, phản ánh các góc nhìn khác nhau về khái niệm "bất thường" trong dữ liệu chuỗi thời gian lưu lượng nước:

\begin{itemize}
    \item \textbf{Phương pháp dựa trên sai số} phát hiện các điểm không tuân theo mẫu thời gian dự đoán.
    \item \textbf{Isolation Forest} phát hiện các điểm dễ bị cô lập, thường là các giá trị cực trị.
    \item \textbf{LOF} phát hiện các điểm có mật độ cục bộ khác biệt so với láng giềng.
    \item \textbf{One-class SVM} phát hiện các điểm nằm ngoài ranh giới của dữ liệu bình thường trong không gian đặc trưng.
\end{itemize}

Việc kết hợp kết quả từ nhiều phương pháp có thể cung cấp cái nhìn toàn diện hơn về các bất thường trong hệ thống cấp nước, tận dụng điểm mạnh của từng phương pháp để phát hiện các dạng bất thường khác nhau. Đặc biệt, việc kết hợp phương pháp dựa trên mô hình (như LSTM) với các phương pháp học không giám sát (như Isolation Forest, LOF và One-class SVM) có thể tạo ra một hệ thống phát hiện bất thường mạnh mẽ và toàn diện hơn cho các ứng dụng thực tế.
