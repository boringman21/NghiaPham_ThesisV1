\begin{tabular}
{p{0.4\textwidth} p{0.55\textwidth}}
     \begin{center}
    \fontsize{11}{15}\selectfont ĐẠI HỌC QUỐC GIA TP.HCM\\
         \fontsize{11}{15}\selectfont\textbf{TRƯỜNG ĐẠI HỌC BÁCH KHOA}\\
         -----------------------------
     \end{center}& \begin{center}
         \fontsize{11}{15}\selectfont\textbf{CỘNG HOÀ XÃ HỘI CHỦ NGHĨA VIỆT NAM}\\
         \fontsize{11}{15}\selectfont\textbf{Độc lập - Tự do - Hạnh phúc}\\
         -----------------------------
     \end{center}  
\end{tabular}

\begin{center}
    \fontsize{16}{20}\selectfont
    \textbf{NHIỆM VỤ LUẬN VĂN THẠC SĨ}

\end{center}
\begingroup
\renewcommand{\arraystretch}{1.5}
\begin{table}[h]
    \centering
    \begin{tabular}{p{0.5\textwidth}p{0.4\textwidth}}
        Họ tên học viên: Trần Hoàng Việt&MSHV: 2270675  \\
        Ngày, tháng, năm sinh: 14/04/2000 &Nơi sinh: thành phố Hồ Chí Minh  \\
        Chuyên ngành: Khoa học máy tính &Mã số: 8480101
    \end{tabular}
\end{table}
\endgroup
\subsubsection*{I. TÊN ĐỀ TÀI:}
\onehalfspacing
Ứng dụng học máy trong phát hiện rò rỉ hệ thống cấp nước (Application of machine learning in water supply system leak detection)

\subsubsection*{II. NHIỆM VỤ VÀ NỘI DUNG}
\begin{itemize}
    \item Nghiên cứu các phương pháp xử lý dữ liệu bị thiếu
    \item Tìm hiểu, phân tích và đề xuất các phương pháp xử lý dữ liệu bị thiếu phù hợp cho dữ liệu chuỗi thời gian, cụ thể là dữ liệu từ các trạm đo quan trắc và nguồn dữ liệu đã được trích xuất thông tin từ ảnh viễn thám
    \item Thực nghiệm huấn luyện các mô hình dự đoán chất lượng không khí với tập dữ liệu đã được bổ khuyết
    \item Đánh giá mức độ phù hợp của các phương pháp lên quá trình bổ khuyết dữ liệu chất lượng không khí
    \item Đánh giá mức độ phù hợp của các phương pháp lên chất lượng của các mô hình dự đoán chất lượng không khí
    \item Đề xuất phương pháp phù hợp để xử lý dữ liệu cho bài toán dự đoán chất lượng không khí
\end{itemize}

\subsubsection*{III. NGÀY GIAO NHIỆM VỤ:} 09/09/2024

\subsubsection*{IV: NGÀY HOÀN THÀNH NHIỆM VỤ:} 23/12/2024

\subsubsection*{V. CÁN BỘ HƯỚNG DẪN:} TS. Lê Thanh Vân

\vspace{1em}
\begin{tabular}
{p{0.4\textwidth} p{0.5\textwidth}}
\vspace{1.25em}
     \begin{center}
         \textbf{CÁN BỘ HƯỚNG DẪN}
     \end{center}& \begin{center}
     Tp. HCM, ngày 20 tháng 02 năm 2025
     \\
         \textbf{HỘI ĐỒNG NGÀNH}
     \end{center}   \\
\end{tabular}
\vspace{5em}
\begin{center}
    \textbf{TRƯỞNG KHOA KHOA HỌC VÀ KỸ THUẬT MÁY TÍNH}
\end{center}
