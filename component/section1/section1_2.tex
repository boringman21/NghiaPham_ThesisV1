\section{Tính cấp thiết của đề tài}
\subsection{Bối cảnh thực tế và thách thức trong quản lý nước}
Hiện nay, các đô thị lớn trên thế giới, đặc biệt là ở các quốc gia đang phát triển, đang đối mặt với áp lực ngày càng gia tăng về nhu cầu sử dụng nước sạch. Theo thống kê của Tổ chức Y tế Thế giới (WHO), hơn 2,1 tỷ người trên thế giới không được tiếp cận đầy đủ với nguồn nước sạch và an toàn\cite{WHO_JMP_2023}. Điều này không chỉ là một vấn đề về sức khỏe cộng đồng mà còn đặt ra thách thức nghiêm trọng trong việc quản lý tài nguyên nước.

Tại Việt Nam, với tốc độ đô thị hóa nhanh chóng, nhu cầu nước sạch tăng trưởng trung bình từ 5\% đến 10\% mỗi năm\cite{MOC_NhuCauNuocSinhHoat}. Tuy nhiên, tỷ lệ thất thoát nước cao tại các đô thị lớn đã làm suy giảm hiệu quả sử dụng tài nguyên nước, đồng thời làm gia tăng chi phí vận hành và giá thành nước sạch. Điều này không chỉ ảnh hưởng đến nền kinh tế mà còn tác động tiêu cực đến môi trường và đời sống xã hội.

\subsection{Lý do chọn đề tài}
Việc nghiên cứu và triển khai các phương pháp hiện đại để phát hiện và ngăn chặn rò rỉ nước mang lại nhiều giá trị thiết thực:
\begin{itemize}
    \item \textbf{Giảm thiểu thất thoát nước:} Phát hiện sớm các điểm rò rỉ giúp giảm lượng nước thất thoát, từ đó tối ưu hóa hiệu quả sử dụng tài nguyên nước.
    \item \textbf{Tăng hiệu quả vận hành:} Các giải pháp công nghệ cao giúp đơn giản hóa quá trình giám sát và bảo trì, giảm thiểu thời gian và chi phí vận hành.
    \item \textbf{Bảo vệ môi trường:} Giảm thất thoát nước đồng nghĩa với việc giảm khai thác nguồn nước thiên nhiên, từ đó giảm áp lực lên các hệ sinh thái.
    \item \textbf{Ứng dụng công nghệ hiện đại:} Việc sử dụng trí tuệ nhân tạo (AI), học máy (Machine Learning) không chỉ nâng cao độ chính xác mà còn thể hiện xu hướng tất yếu trong thời đại 4.0.
    \item \textbf{Giải quyết vấn đề cấp bách:} Với tỷ lệ thất thoát nước cao tại Việt Nam, đề tài mang tính ứng dụng thực tiễn cao, góp phần cải thiện chất lượng dịch vụ và nâng cao hiệu quả quản lý tài nguyên.
\end{itemize}

\subsection{Tầm quan trọng khoa học và thực tiễn}
Đề tài không chỉ có giá trị về mặt khoa học mà còn đáp ứng nhu cầu cấp thiết trong thực tế. Các mô hình AI và học máy, đang ngày càng được ứng dụng rộng rãi trong việc giải quyết các vấn đề phức tạp. Với khả năng phân tích dữ liệu đa chiều và học hỏi từ các mẫu phức tạp, các mô hình này hứa hẹn sẽ mang lại những bước tiến lớn trong lĩnh vực phát hiện và ngăn chặn rò rỉ nước.
