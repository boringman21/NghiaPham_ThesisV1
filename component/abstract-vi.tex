\chapter*{Tóm tắt luận văn thạc sĩ}
Cùng với sự phát triển của thời kỳ công nghiệp 4.0, ngày càng nhiều thiết bị IOT ra đời, đã và đang mang lại một lượng dữ liệu khổng lồ. Từ đó, nhu cầu khai thác các thông tin ẩn nhưng giá trị từ nguồn dữ liệu này là vô cùng cấp thiết. Tuy nhiên, việc mất mát dữ liệu là vấn dề không thể tránh khỏi, nhất là dữ liệu lấy từ các thiết bị, cảm biến IOT trong hệ thống giám sát cấp nước. Các mất mát này có thể đến từ nhiều nguyên nhân khác nhau như cảm biến gặp lỗi, sự cố mất điện, sự cố khi truyền tải dữ liệu, hoặc trạm tiếp nhận dữ liệu gặp trục trặc. Điều này ảnh hưởng không nhỏ đến chất lượng của các mô hình phân tích và dự đoán rò rỉ nước. Do đó, việc xử lý dữ liệu bị thiếu đang trở thành chủ đề đáng chú ý của các nhà khoa học hiện nay. Phương pháp xử lý dữ liệu bị thiếu phù hợp không chỉ giúp giữ lại được các thông tin quan trọng của dữ liệu, mà nó còn góp phần cải thiện độ chính xác của mô hình phát hiện rò rỉ nước.

Và khi xét đến các lĩnh vực nghiên cứu đang được quan tâm, bài toán phát hiện rò rỉ nước là một trong những bài toán không thể không nhắc đến. Nhiều năm trở lại đây, tình trạng thất thoát nước ở Việt Nam luôn ở trong mức báo động, đặc biệt tại các đô thị lớn như thủ đô Hà Nội, thành phố Đà Nẵng và thành phố Hồ Chí Minh. Tỷ lệ thất thoát nước cao gây ra tác động không nhỏ đến nguồn tài nguyên nước, chi phí vận hành và duy tu bảo dưỡng hệ thống, dẫn đến các thiệt hại về kinh tế và sự phát triển bền vững. Rò rỉ nước có thể xuất phát từ nhiều nguyên nhân như đường ống già cỗi, áp lực nước không ổn định, lắp đặt không đúng kỹ thuật, và thiên tai. Tình trạng rò rỉ nước làm tăng chi phí sản xuất, giảm hiệu quả cung cấp nước, và trong một số trường hợp có thể gây ngập úng, sụt lún đất. Nhằm chủ động phát hiện, giảm thiểu thấp nhất tác động của rò rỉ nước đến hệ thống cấp nước và môi trường, việc xây dựng một hệ thống phát hiện sớm rò rỉ nước là vô cùng quan trọng. Thông qua đó, các nhà quản lý có thể nhanh chóng ứng phó và khắc phục sự cố, tiết kiệm nguồn nước và chi phí vận hành.

Nghiên cứu này tập trung vào cải thiện chất lượng của các mô hình phát hiện rò rỉ nước thông qua phân tích và bổ khuyết dữ liệu bị thiếu từ nhiều nguồn. Báo cáo nêu lên tầm quan trọng của tiền xử lý dữ liệu trong việc nâng cao độ chính xác và tin cậy cho mô hình phát hiện. Các phương pháp đã được nghiên cứu bao gồm phương pháp cổ điển, phương pháp sử dụng học sâu, phương pháp sử dụng đơn thuộc tính và phương pháp kết hợp nhiều thuộc tính. Ngoài ra, tác giả còn thử nghiệm tỉ lệ mất dữ liệu ảnh hưởng như thế nào đến các phương pháp phát hiện rò rỉ nước. Cuối cùng, nghiên cứu kiểm nghiệm mức độ cải thiện hiệu quả mô hình phát hiện rò rỉ với dữ liệu đã tiền xử lý.