\section{Định hướng nghiên cứu} 
\subsection{Mục tiêu nghiên cứu}
Mục tiêu nghiên cứu của đề tài là tìm hiểu và phân tích và đề xuất các phương pháp xử lý dữ liệu bị thiếu phù hợp cho dữ liệu chuỗi thời gian, cụ thể là dữ liệu từ các trạm đo quan trắc và nguồn dữ liệu đã được trích xuất thông tin từ ảnh viễn thám. Kết quả xử lý dữ liệu bị thiếu sẽ được đưa vào vào huấn luyện các mô hình để giúp nâng cao hiệu quả dự đoán chất lượng không khí. Ngoài ra, nghiên cứu này có thể được mở rộng để áp dụng cho các bài toán liên quan đến dữ liệu thời gian và đa thuộc tính khác.

\subsection{Nhiệm vụ nghiên cứu}
Đề tài tập trung vào các nhiệm vụ sau:
\begin{itemize}
    \item Nghiên cứu các phương pháp xử lý dữ liệu bị thiếu
    \item Phân tích và tiền xử lý dữ liệu tập dữ liệu về chất lượng không khí
     
    \item Thực nghiệm các mô hình huấn luyện khác nhau với dữ liệu sau khi đã được xử lý để so sánh và đánh giá tổng quan về mức độ phù hợp của từng mô hình 
    \item Từ kết quả thực nghiệm, đề xuất phương pháp phù hợp để xử lý dữ liệu cho bài toán dự đoán chất lượng không khí
\end{itemize}

\subsection{Đối tượng nghiên cứu}
Đối tượng nghiên cứu của đề tài bao gồm dữ liệu từ quan trắc và viễn thám. Các dữ liệu được thu thập từ vệ tinh và 6 trạm đo đạc ở thành phố Hồ Chí Minh trong năm 2021 và 2022. 


\subsection{Phạm vi nghiên cứu}
Đề tài được giới hạn về đối tượng và phạm vi nghiên cứu như sau:
\begin{itemize}
    \item Bổ khuyết và dự đoán các chỉ số không khí: bụi $TSP$, bụi $PM_{2.5}$, $O_3$, $CO$, $NO_2$, $SO_2$.
    \item Các mất mát dữ liệu đến từ các sự cố do máy móc, mang tính ngẫu nhiên
\end{itemize}