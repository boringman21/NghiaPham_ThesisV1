\chapter{Tổng quan}
Những năm gần đây, nhiều nhà khoa học đã thực hiện nhiều thí nghiệm nhằm nâng cao hiệu quả của các mô hình dự đoán chất lượng không khí. Bên cạnh các mô hình xây dựng các công thức tính toán phức tạp và nặng nề, các mô hình theo hướng thống kê, học máy và học sâu ngày càng được nghiên cứu sâu hơn. Ưu điểm của các mô hình này là có thể tự học được các đặc trưng ẩn từ dữ liệu. Từ đó, các mô hình này sẽ giúp giảm chi phí tính toán và cho kết quả dự đoán tốt hơn. Có 3 loại mô hình thường được áp dụng là: các mô hình học máy, các mô hình học sâu và các mô hình kết hợp giữa các phương pháp.

Các nghiên cứu áp dụng mô hình học máy có thể kể đến là \textit{Prediction of air pollution index (API) using support vector machine (SVM)} \cite{LEONG2020103208}. Các tác giả đã sử dụng support vector machine kết hợp với multiple linear regression để dự đoán chỉ số ô nhiễm không khí. Đặc biệt, bài báo chỉ ra rằng dữ liệu sau khi được tiền xử lý bằng Interquartile range (IQR) đã giúp cải thiện chất lượng mô hình dự đoán (sum square error giảm đi 10 lần). Tiếp theo, trong bài báo về dự đoán chỉ số $PM_{2.5}$ ở thành phố Hồ Chí Minh của Rajnish et al. \cite{paper_bang}, các tác giả đã sử dụng phương pháp tiền xử lý dữ liệu moving average trước khi huấn luyện các mô hình. Điều này giúp giá trị metric của các mô hình dùng XGBoost và SGD regressor giảm đi 2 lần.

Bên cạnh các mô hình học máy, các mô hình học sâu ngày nay cũng trở thành tâm điểm của các nhà nghiên cứu, và đã đạt được những bước tiến đáng kể. Các mô hình này có khả năng tự học các đặc trưng ẩn của dữ liệu. Một trong những bài báo tiêu biểu theo hướng nghiên cứu này là \textit{Combining forward with recurrent neural networks for hourly air quality prediction in Northwest of China} của Zhili Zhao và các cộng sự \cite{ann-rnn}. Nhóm tác giả đã thành công trong việc tạo ra mô hình kết hợp giữa phương pháp học máy và học sâu để đưa ra giá trị metric MAE cực kỳ thấp. Để đạt được kết quả này, bài báo đã áp dụng singular spectrum analysis \cite{article-ssa} để cải thiện chất lượng dữ liệu trước khi huấn luyện.

Ngoài ra, trong các nghiên cứu trong và ngoài nước khác, các tác giả còn vận dụng kết hợp nhiều loại thuộc tính khác nhau trong quá trình tiền xử lý để cải thiện chất lượng mô hình. Một nghiên cứu có thể kể đến là \textit{GAIN: Missing Data Imputation using Generative Adversarial Nets
} của Yoon và các cộng sự \cite{pmlr-v80-yoon18a}. Trong bài báo trên, các tác giả đã sử dụng biến thể của Conditional Generative Adversarial Nets để dự đoán xác suất phân phối của dữ liệu, với đầu vào là dữ liệu bị thiếu và đầu ra của Discriminator là ma trận có kích thước tương đương. Trong ma trận đó, mỗi ô có nhãn là 0 hoặc 1 ám chỉ giá trị của thuộc tính $i$, thời điểm $t$ là do Generator sinh ra hay dữ liệu thật. Bên cạnh đó, bài báo \textit{The impact of data imputation on air quality prediction problem} của Van Hua et al. \cite{article-vanhua} cũng đề cập và so sánh phương pháp Multivariate Imputation by Chained Equations (MICE) với các phương pháp truyền thống khác. Tuy nhiên, các tác giả trong bài báo này chưa xem xét đến việc khởi tạo giá trị ban đầu cho MICE bằng các giá trị khác giá trị trung bình. Đối với các tập dữ liệu bị mất cùng lúc nhiều thuộc tính tại một thời điểm và liên tục, việc sử dụng giá trị trung bình để khởi tạo cho MICE chưa thực sự đủ tốt.

Trong luận văn này, đề tài sẽ tập trung vào các nội dung chính sau:
\begin{itemize}
    \item So sánh các phương pháp bổ khuyết cổ điển và học sâu, các phương pháp sử dụng đơn thuộc tính và kết hợp nhiều thuộc tính
    \item Thử nghiệm phương pháp MICE với các giá trị khởi tạo ban đầu khác nhau
    \item Chọn lọc các thuộc tính có thể kết hợp với nhau một cách phù hợp cho các phương pháp sử dụng kết hợp nhiều thuộc tính
    \item Thử nghiệm các phương pháp bổ khuyết với các tập dữ liệu có tỉ lệ mất mát khác nhau
    \item Đánh giá mức độ ảnh hưởng của các dữ liệu đã tiền xử lý lên chất lượng của các mô hình dự báo chất lượng không khí
\end{itemize}