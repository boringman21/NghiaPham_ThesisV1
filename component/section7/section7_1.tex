\chapter{Kết luận}
\section{Ưu điểm của đề tài}
Đề tài đã nghiên cứu và thử nghiệm các phương pháp bổ khuyết giúp nâng cao chất lượng các mô hình dự đoán chất lượng không khí bao gồm:
\begin{itemize}
    \item Phương pháp cổ điển
    \begin{itemize}
        \item Phương pháp cổ điển trên đơn thuộc tính
        \item Phương pháp cổ điển kết hợp nhiều thuộc tính
    \end{itemize}
    \item Phương pháp học sâu
    \begin{itemize}
        \item Phương pháp học sâu trên đơn thuộc tính
        \item Phương pháp học sâu kết hợp nhiều thuộc tính
    \end{itemize}
\end{itemize}

Đồng thời, đối với phương pháp MICE, tác giả cũng đề xuất các phương pháp khác nhau khi khởi tạo giá trị ban đầu cho MICE giúp nâng cao chất lượng xử lý dữ liệu của phương pháp này. Bên cạnh đó, đề tài còn thử nghiệm trong các trường hợp tỉ lệ mất mát dữ liệu khác nhau, số lượng dữ liệu bị mất tại cùng một thời điểm khác nhau, và đánh giá kết quả khi áp dụng các phương pháp bổ khuyết lên huấn luyện các mô hình dự đoán chất lượng không khí.

...