\chapter*{Tóm tắt luận văn thạc sĩ}
Cùng với sự phát triển của thời kỳ công nghiệp 4.0, ngày càng nhiều thiết bị IOT ra đời, đã và đang mang lại một lượng dữ liệu khổng lồ. Từ đó, nhu cầu khai thác các thông tin ẩn nhưng giá trị từ nguồn dữ liệu này là vô cùng cấp thiết. Tuy nhiên, việc mất mát dữ liệu là vấn dề không thể tránh khỏi, nhất là dữ liệu lấy từ các thiết bị, cảm biến IOT. Các mất mát này có thể đến từ nhiều nguyên nhân khác nhau như cảm biến gặp lỗi, sự cố mất điện, sự cố khi truyền tải dữ liệu, hoặc trạm tiếp nhận dữ liệu gặp trục trặc. Điều này ảnh hưởng không nhỏ đến chất lượng của các mô hình phân tích và dự đoán dữ liệu. Do đó, việc xử lý dữ liệu bị thiếu đang trở thành chủ đề đáng chú ý của các nhà khoa học hiện nay. Phương pháp xử lý dữ liệu bị thiếu phù hợp không chỉ giúp giữ lại được các thông tin quan trọng của dữ liệu, mà nó còn góp phần cải thiện độ chính xác của mô hình dự đoán.

Và khi xét đến các lĩnh vực nghiên cứu đang được quan tâm, bài toán dự đoán chất lượng không khí là một trong những bài toán không thể không nhắc đến. Nhiều năm trở lại đây, tình trạng ô nhiễm không khí ở Việt Nam luôn ở trong mức báo động, nhất là các đô thị lớn như thủ đô Hà Nội, thành phố Đà Nẵng và thành phố Hồ Chí Minh. Chất lượng không khí thấp gây ra tác động không nhỏ đến sức khoẻ của người dân, dẫn đến các thiệt hại về sức khoẻ, tiền bạc và sự phát triển kinh tế xã hội. Không khí bị ô nhiễm chủ yếu có nguồn gốc từ bụi mịn ($PM_{10}$, $PM_{2.5}$, $TSP$) và các loại khí thải từ nhà máy và xe cộ như $SO_2$, $NO_2$, $CO$, $O_3$. Ô nhiễm không khí là tác nhân gia tăng các bệnh về đường hô hấp, bệnh tim mạch, thậm chí là đột quỵ. Nhằm chủ động cảnh báo, giảm thiểu thấp nhất tác động của ô nhiễm không khí đến sức khoẻ con người, việc xây dựng một hệ thống dự báo sớm chỉ số chất lượng không khí là vô cùng quan trọng. Thông qua đó, người dân có thể chuẩn bị các biện pháp phòng chống và hạn chế các ảnh hưởng của chất lượng không khí đến sức khoẻ.

Nghiên cứu này tập trung vào cải thiện chất lượng của các mô hình dự đoán chất lượng không khí thông qua phân tích và bổ khuyết dữ liệu bị thiếu từ nhiều nguồn. Báo cáo nêu lên tầm quan trọng của tiền xử lý dữ liệu trong việc nâng cao độ chính xác và tin cậy cho mô hình dự đoán. Các phương pháp đã được nghiên cứu bao gồm phương pháp cổ điển, phương pháp sử dụng học sâu, phương pháp sử dụng đơn thuộc tính và phương pháp kết hợp nhiều thuộc tính. Ngoài ra, tác giả còn thử nghiệm tỉ lệ mất dữ liệu ảnh hưởng như thế nào đến các phương pháp trên. Cuối cùng, nghiên cứu kiểm nghiệm mức độ cải thiện hiệu quả mô hình với dữ liệu đã tiền xử lý.