\chapter*{Tóm tắt luận văn thạc sĩ}

Trong bối cảnh nhu cầu sử dụng nước sạch ngày càng tăng cùng với áp lực quản lý hiệu quả hệ thống hạ tầng đô thị, việc giám sát và phát hiện rò rỉ trong mạng lưới cấp nước đóng vai trò then chốt trong bảo vệ tài nguyên và nâng cao chất lượng dịch vụ. Rò rỉ nước không chỉ gây thất thoát đáng kể về mặt kinh tế mà còn tiềm ẩn nguy cơ làm suy giảm chất lượng nước, phá vỡ cân bằng vận hành và ảnh hưởng đến sự phát triển bền vững của các đô thị.

Luận văn này tập trung nghiên cứu ứng dụng các kỹ thuật học máy hiện đại nhằm phát hiện rò rỉ nước dựa trên việc phân tích dữ liệu vận hành thực tế của hệ thống cấp nước, đặc biệt là dữ liệu lưu lượng. Trọng tâm nghiên cứu là xây dựng mô hình dự đoán lưu lượng nước trong điều kiện bình thường, từ đó phát hiện các bất thường tiềm tàng có thể liên quan đến hiện tượng rò rỉ.

Các mô hình học sâu như mạng nơ-ron hồi tiếp dài-ngắn hạn (Long Short-Term Memory - LSTM) và mạng nơ-ron hồi tiếp có cổng xoay (Gated Recurrent Unit - GRU) đã được triển khai nhằm học và mô hình hóa chuỗi thời gian lưu lượng nước. Dựa trên sai số dự đoán giữa mô hình và thực tế, các tín hiệu bất thường được xác định làm cơ sở ban đầu cho việc phát hiện rò rỉ.

Ngoài ra, luận văn còn tích hợp và đánh giá các kỹ thuật phát hiện bất thường nổi bật như Isolation Forest, Local Outlier Factor (LOF), và One-Class SVM. Những phương pháp này hoạt động độc lập với mô hình dự đoán và được sử dụng để phát hiện các mẫu dữ liệu dị thường trong không gian đặc trưng, từ đó tăng cường khả năng nhận diện rò rỉ một cách đa chiều và toàn diện.

Kết quả thực nghiệm trên dữ liệu thực tế cho thấy rằng việc kết hợp giữa các mô hình học sâu và thuật toán phát hiện bất thường truyền thống mang lại hiệu quả cao trong việc nhận diện các sự kiện bất thường liên quan đến rò rỉ. Nghiên cứu không chỉ chứng minh tính khả thi và hiệu quả của học máy trong lĩnh vực hạ tầng đô thị mà còn mở ra tiềm năng phát triển các hệ thống giám sát tự động và thông minh phục vụ công tác quản lý tài nguyên nước trong tương lai.
